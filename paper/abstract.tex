\begin{abstract}
	Sistemul de operare Android este cea mai populara platforma pentru telefoanele
	mobile, iar limbajul utilizat pentru a dezvolta aplicații este Java.
	O reducere de doar câțiva octeți a dimensiunii pachetului unei aplicații
	populare, precum Facebook, ar duce la economisirea de câțiva giga-octeți de
	trafic de internet lunar.

	In aceasta teza am proiectat și implementat un compilator care efectuează
	optimizări de dimensiune a codului asupra fișierelor compilate Java.

	Aceste optimizări elimina funcțiile și metodele neutilizate dintr-un proiect
	dezvoltat in limbajul Java și se bazează pe analiza statica a proiectului.
	O serie de teste au fost create pentru a testa corectitudinea, cat și eficienta
	optimizărilor aplicate.
	Compilatorul lucrează direct cu fișiere compilate, in formatul utilizat și de
	către Android.

\end{abstract}
