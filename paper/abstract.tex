\begin{abstract}
	Sistemul de operare Android este cea mai populara platforma pentru telefoanele
	mobile, iar limbajul utilizat pentru a dezvolta aplicatii este Java.
	O reducere de doar cativa octeti a dimensiunii pachetului unei aplicatii
	populare, precum Facebook, ar duce la economisirea de cativa giga-octeti de
	trafic de internet lunar.

	In aceasta teza am proiectat si implementat un compilator care efectueaza
	optimizari de dimensiune a codului asupra fisierelor compilate Java.

	Aceste optimizari elimina functiile si metodele neutilizate dintr-un proiect
	dezvoltat in limbajul Java si se bazeaza pe analiza statica a proiectului.
	O serie de teste au fost create pentru a testa corectitudinea, cat si eficienta
	optimizarilor aplicate.
	Compilatorul lucreaza direct cu fisiere compilate, in formatul utilizat si de
	catre Android.

\end{abstract}
