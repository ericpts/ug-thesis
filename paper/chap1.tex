\chapter{Detalii despre Java}

\section{Limbajul Java}

Java este un limbaj de programare orientat pe obiecte. Acesta a fost
dezvoltat de catre Sun Microsystems ( acum Oracle), iar prima versiune a
aparut in anul 1995.

Jaza s-a bazat pe sintaxa limbajului C, si a introdus notiunea de
``scrie o data, ruleaza peste tot'' (eng. ``write once, run
everywhere''). Spre deosebire de C si de C++, care trebuiesc compilate
pentru fiecare platforma tinta, Java a avut avantajul ca trebuie
compilat o singura data, si va merge garantat pe toate platformele
suportate de limbaj.

\section{Java Bytecode}

Solutia limbajului Java pentru a fi independent de platforma este de
transforma codul intr-o reprezentare intermediara, in loc de direct in
cod binary pentru o anumita arhitectura .

Compilatorul Java (\texttt{javac}), transforma codul Java intr-un limbaj
intermediar, numit Java Bytecode.

Acest limbaj este un limbaj low-level, destinat in mod exclusiv
procesarii de catre masini, spre deosebire de codul Java, care este
destinat oamenilor.

Dupa ce compilatorul a procesat codul Java, provenit din fisere .java in
format text, acesta salveaza rezultatul in fisiere de tip clasa (.class)
in format binar.

\section{JVM}

Odata generate fisierele binare, acestea sunt executate pe o masina
virtuala specifica limbajului Java - numita \texttt{JVM}
(eng. Java Virtual Machine).

Aceasta masina virtuala are rolul de a citi fisierele de clasa binare si
de a le interpreta.

Masina virtuala este implementata ca o ``masina cu stiva'' (eng. stack
machine), unde toate instructiunile limbajului bytecode interactioneaza
cu datele de pe o stiva controlata de aplicatie.

Masina virtuala insusi este implementata in C/C++, si este compilata in
cod binar direct, dependent de arhitectura. Dezvoltatorii limbajului
Java sunt responsabili pentru corectitudinea si siguranta masinii
virtuale, in timp ce dezvoltatorii de aplicatii Java au garantia ca daca
codul lor Java este corect, atunci acesta va rula la fel, deterministic,
pe orice platforma.

In acest regard, limbajul Java poate fi vazut ca un limbaj interpretat.
Comparand cu alte limbaje populare interpretate, ca de exemplu Python,
Ruby, sau Perl, ne-am astepta ca si Java sa fie la fel de incet ca
acestea \cite{language_benchmarks}. Totusi, Java obtine performante mult mai bune decat
aceastea. Acest fapt se datoreaza compilarii tocmai-la-timp (eng.
just-in-time), in care atunci cand interpretorul observa o secventa de
cod care este interpretata repetitiv de foarte multe ori, va genera
direct cod binary pentru aceasta.
