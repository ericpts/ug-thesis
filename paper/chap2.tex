\chapter{Fisierele clasa}

Fisierele de clasa Java sunt formate din 10
sectiuni\cite{classfile_sections}:

\begin{enumerate}
\item
  Constanta magica.
\item
  Versiunea fisierului.
\item
  Constantele clasei.
\item
  Permisiunile de acces.
\item
  Numele clasei din fisier.
\item
  Numele superclasei.
\item
  Interfetele pe care clasa le implementeaza.
\item
  Campurile clasei.
\item
  Metodele clasei.
\item
  Atribute ale clasei.
\end{enumerate}

In continuare voi da o scurta descriere a formatului sectiunilor.

\section{Sectiunile fiserelor clasa}

\subsection{Magic}

Toate fiserele clasa trebuiesc sa inceapa cu un numar denumit constanta
magica. Acesta este folosit pentru a identifica in mod unic ca acestea
sunt intra-devar fisiere clasa. Numarul magic are o valoare memorabila:
reprezentarea hexadecimala este \texttt{0xCAFEBABE},

\subsection{Versiunea}

Versiunea unui fisier clasa este data de doua valori, versiunea majora
\texttt{M} si versiunea minora \texttt{m}. Versiunea clasei este atunci
reprezentata ca \texttt{M.m}. (e.g., \texttt{45.1}). Aceasta este
folosita pentru a mentine compatibilitatea in cazul modificarilor
masinii virtuale care interpreteaza clasa sau ale compilatorului care o
genereaza.

\subsection{Constantele clasei}

Tabela de constante este locul unde sunt stocate valorile literale
constante ale clasei:

\begin{itemize}
    \item Numere intregi.
    \item Numere cu virgula mobula.
    \item Siruri de caractere, care pot reprezenta la randul lor:
        \begin{itemize}
            \item Nume de clase.
            \item Nume de metode.
            \item Tipuri ale metodelor.
        \end{itemize}
    \item Informatii compuse din datele anterioare:
        \begin{itemize]}
            \item Referinta la o metoda a unei clase.
            \item Referinta la o constanta a unei clase.
        \end{itemize]}
\end{itemize}

Toate celelalte tipuri de date compuse, cum ar fi metodele sau
campurile, vor contine indecsi in tabela de constante.

\subsection{Permisiunile de acces}

Aceste permisiuni constau intr-o masca de bitsi, care reprezeinta
operatiile permise pe aceasta clasa:

\begin{itemize}
    \item daca clasa este publica, si poate fi accesta din afara pachetului acesteia.
    \item daca clasa este finala, si daca poate fi extinsa.
    \item daca invocarea metodelor din superclasa sa fie tratata special.
    \item daca este de fapt o interfata, si nu o clasa.
    \item daca este o clasa abstracta si nu poate fi instatiata.
\end{itemize}

\subsection{Clasa curenta}

Reprezinta un indice in tabela de constante, unde sunt stocate
informatii despre clasa curenta.

\subsection{Clasa super}

Reprezinta un indice in tabela de consatante, cu informatii despre clasa
din care a mostenit clasa curenta. Daca este 0, inseamna ca clasa
curenta nu mosteneste nimic: singura clasa fara o superclasa este clasa
Object.

E.g. pentru

\begin{lstlisting}[language=Java]
public class MyClass extends S implements I
\end{lstlisting}

Indicele corespunde lui \texttt{S}.

\subsection{Interfetele}

Reprezinta o colectie de indici in tabela de constante. Fiecare valoare
de la acei indici reprezinta o interfata implementata in mod direct de
catre clasa curenta. Interfetele apar in ordinea declarata in fisierele
java.

E.g. pentru

\begin{lstlisting}[language=Java]
class MyClass extends S implements I1, I2
\end{lstlisting}

Primul indice ar corespunde lui \texttt{I1}, iar al doilea lui
\texttt{I2}.

\subsection{Campurile}\label{campurile}

Reprezinta informatii despre campurile (eng. fields) clasei:
\begin{itemize}
    \item Permisiunile de acces: daca este public sau privat, etc.
    \item Numele campului.
    \item Tipul campului.
    \item Alte atribute: daca este deprecat, daca are o valoare constanta, etc.
\end{itemize}

\subsection{Metodele}\label{metodele}

Reprezinta informtii despre toate metodele clasei, si include si
constructorii:

\begin{itemize}
    \item Permisiuni de acces: daca este public sau privat, daca este finala, daca este abstracta.
    \item Numele metodei.
    \item Tipul metodei.
    \item In caz ca nu este abstracta, byte codul metodei.
    \item Alte atribute:
        \begin{itemize}
            \item Ce exceptii poate arunca.
            \item Daca este deprecata.
        \end{itemize}
\end{itemize}

Codul metodei este partea cea mai importanta, iar formatul acestuia
urmeaza sa fie detaliat ulterior.

\subsection{Atributele}

Reprezinta alte informatii despre clasa, cum ar fi:
\begin{itemize}
    \item Clasele definite in interiorul acesteia.
    \item In caz ca este o clasa anonima sau definita local, metoda in care este definita.
    \item Numele fisierul sursa din care a fost compilata clasa.
\end{itemize}

In continuare, voi descrie din punct de vedere tehnic tipurile de date
intalnite in fisierele de clasa:

\subsection{Tipurile de baza}

In formatul fisierelor clasa exista trei tipuri de baza, toate bazate pe
intregi. In caz ca un intreg are mai multi octeti, acestia au ordinea de
\texttt{big-endian}: cel mai semnificativ octet va fi mereu primul in
memorie.

\begin{longtable}[]{@{}ccc@{}}
\toprule
Nume & Semantica & Echivalentul in C\tabularnewline
\midrule
\endhead
\texttt{u1} & intreg pe un octet, fara semn & \texttt{unsigned\ char}
sau \texttt{uint8\_t}\tabularnewline
\texttt{u2} & intreg pe doi octeti, fara semn & \texttt{unsigned\ short}
sau \texttt{uint16\_t}\tabularnewline
\texttt{u4} & intreg pe un octet, fara semn & \texttt{unsigned\ int} sau
\texttt{uint32\_t}\tabularnewline
\bottomrule
\end{longtable}

In codul sursa al proiectului, acestea sunt tratate astfel:

\begin{lstlisting}[language=C++]
using u1 = uint8_t;
using u2 = uint16_t;
using u4 = uint32_t;
\end{lstlisting}

\subsection{Tipuri de date compuse}

\subsubsection{cp\_info}

Fiecare constanta din tabela de constante incepe cu o eticheta de 1
octet, care reprezinta datele si tipul structurii. Continutul acesteia
variaza in functie de eticheta, insa indiferent de eticheta, continutul
trebuie sa aiba cel putin 2 octeti.

Aproape toate tipurile de constante ocupa un singur slot in tabela.
Din motive istorice, unele constante ocupa doua sloturi.

Totodata, tot din motive istorice, tabela este indexata de la 1, si nu
de la 0, cum sunt celelalte.

\subparagraph{Tipurile de constante}\label{tipurile-de-constante}

\texttt{CONSTANT\_Class}

Corespunde valorii etichetei de 7 si contine un indice spre un alt camp
in tabela de constante, de tipul \texttt{CONSTANT\_Utf8} - un sir de
caractere. Acel sir de caractere va contine numele clasei.

\texttt{CONSTANT\_Fieldref}

Corespunde valorii etichetei de 9 si contine o referinta spre campul
unei clase. Referinta conta in doi indici, amandoi care arata spre
tabela de contante. Primul indice arata spre o constanta
\texttt{CONSTANT\_Class}, care reprezinta clasa sau interfata careia
apartine metoda. Al doilea indice arata spre o constanta
\texttt{CONSTANT\_NameAndType}, care contine informatii despre numele si
tipul campului.

\texttt{CONSTANT\_Methodref}

Corespunde valorii etichetei de 10 si contine o referinta spre metoda
unei clase. Are o structura identica cu \texttt{CONSTANT\_Fieldref},
doar ca primul indice arata neaparat spre o clasa, in timp ce al doilea
indice arata spre numele si tipul metodei.

\texttt{CONSTANT\_InterfaceMethodref}

Corespunde valorii etichetei de 11 si contine o refereinta spre metoda
unei interfete. Are o structura identica cu
\texttt{CONSTANT\_Methodref}, doar ca primul indice arata spre o
interfata.

\texttt{CONSTANT\_String}

Corespunde valorii etichetei de 8 si reprezinta un sir de caractere.
Contine un indice, catre o structura de tipul \texttt{CONSTANT\_Utf8}.

\texttt{CONSTANT\_Integer}

Corespunde valorii etichetei de 3 si contine un intreg pe 4 octeti.

\texttt{CONSTANT\_Float}

Corespunde valorii etichetei de 4 si contine un numar cu virgula mobila
pe 4 octeti.

\texttt{CONSTANT\_Long}

Corespunde valorii etichetei de 5 si contine un intreg pe 8 octeti. Din
motive istorice, ocupa 2 spatii in tabela de constante.

\texttt{CONSTANT\_Double}

Corespunde valorii etichetei de 6 si contine un numar cu virgula mobila
pe 8 octeti. Din motive istorice, ocupa 2 spatii in tabela de constante.

\texttt{CONSTANT\_NameAndType}

Corespunde valorii etichetei de 12. Descrie numele si tipul unui camp
sau al unei metode, fara informatii despre clasa. Contine doi indici,
amandoi catre structuri de tipul \texttt{CONSTANT\_Utf8}. Primul
reprezinta numele, iar al doilea tipul.

\texttt{CONSTANT\_Utf8}

Corespunde valorii etichetei de 1. Reprezinta un sir de caractere
encodat in formatul UTF-8. Contine un intreg \texttt{length}, de tipul
\texttt{u2}, si apoi \texttt{length} octeti care descriu sirul in sine.
Din cauza ca este encodat ca UTF-8, un singur caracter poate fi format
din mai multi octeti.

\texttt{CONSTANT\_MethodHandle}

Corespunde valorii etichetei de 15 si contine o referinte catre un camp,
o metoda de clasa, sau o metoda de interfata.

\texttt{CONSTANT\_MethodType}

Corespunde valorii etichetei de 16 si contine un indice catre o
constanta \texttt{CONSTANT\_UTf8}, ce reprezinta tipul unei metode.

\texttt{CONSTANT\_InvokeDynamic}

Corespunde valorii etichetei de 18 si este folosit de catre \texttt{JVM}
pentru a invoka o metoda polimorfica.

In cod \texttt{C++}, am reprezentat \texttt{cp\_info} astfel:

\begin{lstlisting}[language=C++]
struct cp_info {
    enum class Tag : u1 {
        CONSTANT_Class = 7,
        CONSTANT_Fieldref = 9,
        CONSTANT_Methodref = 10,
        CONSTANT_InterfaceMethodref = 11,
        CONSTANT_String = 8,
        CONSTANT_Integer = 3,
        CONSTANT_Float = 4,
        CONSTANT_Long = 5,
        CONSTANT_Double = 6,
        CONSTANT_NameAndType = 12,
        CONSTANT_Utf8_info = 1,
        CONSTANT_MethodHandle = 15,
        CONSTANT_MethodType = 16,
        CONSTANT_InvokeDynamic = 18,
    };

    Tag tag;
    std::vector<u1> data;
};
\end{lstlisting}

Iar structurile folosite pentru obiectivul propus au fost reprezentate
astfel:

\begin{lstlisting}[language=C++]
struct CONSTANT_Methodref_info {
    cp_info::Tag tag;
    u2 class_index;
    u2 name_and_type_index;
};
struct CONSTANT_Class_info {
    cp_info::Tag tag;
    u2 name_index;
};
struct CONSTANT_NameAndType_info {
    cp_info::Tag tag;
    u2 name_index;
    u2 descriptor_index;
};
\end{lstlisting}

\paragraph{\texorpdfstring{\texttt{field\_info}}{field\_info}}\label{field_info}

Fiecare camp din cadrul unei clase este reprezentat printr-o structura
de tipul \texttt{field\_info}.

In cod \texttt{C++}, aceasta structura a fost reprezentata astfel:

\begin{lstlisting}[language=C++]
struct field_info {
    u2 access_flags;
    u2 name_index;
    u2 descriptor_index;
    u2 attributes_count;
    std::vector<attribute_info> attributes;
};
\end{lstlisting}

Unde:
\begin{itemize}
    \item \texttt{name\_index} este o intrare in tabela de constante unde se afla o constanta de tipul \texttt{CONSTANT\_Utf8}.
    \item \texttt{descriptor\_index} arata spre o constanta de tipul \texttt{CONSTANT\_Utf8} si reprezinta tipul campului.
\end{itemize}

\paragraph{\texorpdfstring{\texttt{method\_info}}{method\_info}}\label{method_info}

Fiecare metoda a unei clase/interfete este descrisa prin aceasta
structura.

In cod \texttt{C++}, am implementat-o asa:

\begin{lstlisting}[language=C++]
struct method_info {
    u2 access_flags;
    u2 name_index;
    u2 descriptor_index;
    u2 attributes_count;
    std::vector<attribute_info> attributes;
};
\end{lstlisting}

Unde \texttt{name\_index} si \texttt{descriptor\_index} au aceeasi
interpretare ca si la \texttt{field\_info}.

Daca metoda nu este abstracta, atunci in vectorul \texttt{attributes} se
va gasi un attribut de tipul \texttt{Code}, care contine bytecode-ul
corespunzator acestei metode.

\paragraph{\texorpdfstring{\texttt{attribute\_info}}{attribute\_info}}\label{attribute_info}

In \texttt{C++}, a fost implementata astfel:

\begin{lstlisting}[language=C++]
struct attribute_info {
    u2 attribute_name_index;
    u4 attribute_length;
    std::vector<u1> info;
};
\end{lstlisting}

Numele atributului determina modul in care octetii din vectorul
\texttt{info} sunt interpretati. Pentru intentiile noastre, atributul de
interes este cel de cod:

\subparagraph{\texorpdfstring{\texttt{Code\_attribute}}{Code\_attribute}}\label{code_attribute}

\begin{lstlisting}[language=C++]
struct Code_attribute {
    u2 attribute_name_index;
    u4 attribute_length;

    u2 max_stack;
    u2 max_locals;

    u4 code_length;
    std::vector<u1> code;

    u2 exception_table_length;
    struct exception {
        u2 start_pc;
        u2 end_pc;
        u2 handler_pc;
        u2 catch_type;
    };
    std::vector<exception> exception_table; // of length exception_table_length.
    u2 attributes_count;
    std::vector<attribute_info> attributes; // of length attributes_count.
};
\end{lstlisting}

Aceasta structura este piesa centrala a lucrarii. In continuare, o voi
descrie detaliat:

\begin{itemize}
\tightlist
\item
  \texttt{max\_stack}: Reprezinta adancimea maxima a stivei masinii
  virtuale cand aceasta bucata de cod este interpretata.
\item
  \texttt{max\_locals}: Reprezinta numarul maxim de variabile locale
  alocate in acelasi timp cand aceasta bucata de cod este interpretata.
\item
  \texttt{code}: Codul metodei.
\item
  \texttt{exception\_table}: Exceptiile pe care le poate arunca metoda.
\end{itemize}

\texttt{Code}

Vectorul \texttt{code} din cadrul atributului \texttt{Code} reprezinta
bytecode-ul propriu-zis al metodei.

Acest vector contine instructiunile care sunt executate de catre masina
virtuala.

JVM-ul ruleaza ca o masina cu stiva, iar toate instructiunile opereaza
pe aceasta stiva. Reultatul rularii unei instructiuni este modificarea
stivei: scoaterea si adaugarea de elemente in varful acesteia.

Instructiunile au in general formatul \cite{instruction_format}:

\begin{verbatim}
nume_instr
operand1
operand2
...
\end{verbatim}

cu un numar variabil de operanzi, prezenti in mod explicit in vectorul
de \texttt{cod}.

Fiecarui instructiuni ii corespunde un octet, denumit opcode. Fiecare
operand este fie cunoscut la compilare, fie calculat in mod dinamic la
rulare.

Cele mai multe operatii nu au niciun operand dat in mod explicit la
nivelul instructiunii: ele lucreaza doar cu valorile din varful stivei
la momentul executarii codului.

De exemplu:

Instructiunea \texttt{imul} are octetul \texttt{104} sau \texttt{0x68}.
Acestea da pop la doua valori din varful stivei: \texttt{value1} si
\texttt{value2}. Amandoua valorile trebuie sa fie de tipul \texttt{int}.
Rezultatul este inmultirea celor doua valori:
\texttt{result\ =\ value1\ *\ value2}, si este pus in varful stivei.

Dintre cele peste o suta de instructiuni, noi suntem preocupati doar de
5 dintre acestea: cele care au de a face cu invocarea unei metode.

invokedynamic

Format:

\begin{verbatim}
invokedynamic
index1
index2
0
0
\end{verbatim}

Opcode-ul corespunzator acestei instructiuni este \texttt{186} sau
\texttt{0xba}.

Index1 si index2 sunt doi octeti sunt compusi in

\begin{Shaded}
\begin{Highlighting}[]
\NormalTok{index = (index1 << }\DecValTok{8}\NormalTok{) | index2}
\end{Highlighting}
\end{Shaded}

Indicele compus reprezinta o intrare in tabela de constante. La locatia
respectiva trebuie sa se afle o structura de tipul
\texttt{CONSTANT\_MethodHandle}

invokeinterface

Format:

\begin{verbatim}
invokeinterface
index1
index2
count
0
\end{verbatim}

Opcode-ul corespunzator este \texttt{185} sau \texttt{0xb9}.
\texttt{index1} si \texttt{index2} sunt folositi, in mod similar ca la
\texttt{invokedynamic}, pentru a construi un \texttt{indice} in tabela
de constante.

La pozitia respectiva in tabela, trebuie sa se regaseasca o strutura de
tipul \texttt{CONSTANT\_Methodref}.

\texttt{count} trebuie sa fie un octet fara semn diferit de 0. Acest
operand descrie numarul argumentelor metodei, si este necesar din motive
istorice: aceasta informatie poate fi dedusa din tipul metodei.

TODO(ericpts): add resolution order.

invokespecial

Format:

\begin{verbatim}
invokespecial
index1
index2
\end{verbatim}

Opcode-u corespunzator este \texttt{183} sau \texttt{0xb7}. La fel ca la
\texttt{invokeinterface}, este format un indice in tabela de constante,
catre o structura \texttt{CONSTANT\_Methodref}.

Aceasta instructiune este folosita pentru a invoca constructorii
claselor.

invokestatic

Format:

\begin{verbatim}
invokestatic
index1
index1
\end{verbatim}

Opcode-ul corespunzator este \texttt{184} sau \texttt{0xb8}.
Instructiunea este invocata pentru a invoke o metoda statica a unei
clase.

La fel ca la \texttt{invokeinterface}, este construit un indice compus,
si folosit pentru a indexa tabela de constante.

invokevirtual

Format:

\begin{verbatim}
invokevirtual
index1
index1
\end{verbatim}

Opcode-ul corespunzator este \texttt{182} sau \texttt{0xb6}, iar
interpretarea este la fel ca la \texttt{invokeinterface}.

Aceasta este cea mai comuna instructiune de invocare de functii.

Dupa ce numele si tipul metodei, cat si clasa \texttt{C} de care
apartine aceasta sunt rezolvate, masina virtuala cauta metoda respectiva
in clasa referentiata. In caz ca o gaseste, cautarea se termina. In caz
negativ, JVM va continua cautarea recursiv din superclasa lui
\texttt{C}.

In \texttt{C++}, am reprezentat aceste instructiuni de interes astfel:

\begin{lstlisting}[language=C++]
enum class Instr {
    invokedynamic = 0xba,
    invokeinterface = 0xb9,
    invokespecial = 0xb7,
    invokestatic = 0xb8,
    invokevirtual = 0xb6,
};
\end{lstlisting}

\subsection{ClassFile}\label{classfile}

Folosind definitiile anterioare, putem descrie un fisier de clasa binar
in C++:


\begin{lstlisting}[language=C++]
struct ClassFile {
    u4 magic; // Should be 0xCAFEBABE.

    u2 minor_version;
    u2 major_version;

    u2 constant_pool_count;
    std::vector<cp_info> constant_pool;

    u2 access_flags;

    u2 this_class;
    u2 super_class;

    u2 interface_count;
    std::vector<interface_info> interfaces;

    u2 field_count;
    std::vector<field_info> fields;

    u2 method_count;
    std::vector<method_info> methods;

    u2 attribute_count;
    std::vector<attribute_info> attributes;
};
\end{lstlisting}

\section{Studiu de caz}

Pentru a vedea un fisier clasa analizat in detaliu, uitati-va la appendix-ul studiu de caz.
