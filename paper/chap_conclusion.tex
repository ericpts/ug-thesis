\chapter{Aplicație și concluzii}

În acest capitol voi prezenta aplicația și concluziile lucrării.

\section{Aplicația}

Programul este implementat în limbajul C++.
Acesta funcționează în mod corect: formează graful de
apelări ale metodelor și elimină metodele nefolosite.

\subsection{Rulare}

Programul vine cu opțiunea --help, care specifică modul de utilizare.
\begin{lstlisting}[language=Bash]

$ thesis --help
JVM Optimizer
Usage: thesis [OPTIONS] classfiles...

Positionals:
  classfiles [File] ... (REQUIRED)
                              All of the class files from the project

Options:
  -h,--help                   Print this help message and exit
  --classfiles [File] ... (REQUIRED)
                              All of the class files from the project
  --out Dir Excludes: --in-place
                              Where to save the modified class files
  --in-place Excludes: --out  Whether to save the class files in-place

\end{lstlisting}

Utilizarea normală ar arăta așa:

\begin{lstlisting}[language=Bash]
$ thesis *.class --in-place
Done parsing constants.
...
Found the main file and method.
...
Trying to resolve method from symbolic reference: (bar) and type (LMain$One;)V.
Trying to resolve_classfile for Main
Successfully resolved bar :: (LMain$One;)V
...
Main/bar :: (LMain$One;)V and Other/foo :: ()V are not sibling methods
Found the following sibling methods for Main/bar :: (LMain$One;)V: Main/bar :: (LMain$One;)V;
...
Successfully resolved foo :: ()V
...
Found 1 method(s) to remove from class Other
Removing Other/foo :: ()V...
Done putting constants.
...
Done putting attributs.
\end{lstlisting}

\section{Testare}

Testarea aplicației este realizată automat, cu teste scrise de mână.
Aceste teste simulează atât cazuri particulare, cât și funcționarea uzuală a
programului.

Teste pot fi rulate prin programul \texttt{test.py}, din rădăcina proiectului:

\begin{lstlisting}[language=Bash]
$ ./test.py
Building...
Running on test/fixtures/project2
...
Running on test/fixtures/project3
Testing on /home/ericpts/work/ug-thesis/test/fixtures/TAP/tema4/var33...
...
Testing on /home/ericpts/work/ug-thesis/test/fixtures/TAP/tema3/Var3Munte...
\end{lstlisting}

\section{Concluzii}

Scopul lucrării a fost de a prezenta o optimizare pentru dimensiunea
programelor - eliminarea funcțiilor nefolosite.
Prima aprte a ei a implicat formalizarea problemei, definirea unui algoritm
generic pentru optimizări de dimensiune și demonstrația
corectitudinii acestuia.

în a doua parte, am prezentat cum algortimul poate fi implementat în limbajul
Java, și am analizat condițiile necesare pentru a putea afirma faptul că
algoritmul poate fi aplicat în mod corect.

În concluzie, implementarea acestui fel de optimizări este un procedeu complex,
cu multe greutăți de implementare.
De asemenea, optimizările prezintă și o sporită dificultate teoretică, în a
identifica asumpțiile făcute de către optimizator, a demonstra suficiența
acestora, și a le prezenta utilizatorilor programului într-un mod ușor de
înțeles.

\section{Îmbunătățiri propuse}

\subsection{Testarea}
Testarea programului ar putea fi îmbunătățită extinzând suita de teste, cu mai
multe cazuri și situații.

De asemenea, programul ar putea fi pus să optimizeze proiecte mai complexe,
pentru verificarea corectitudinii.

\subsection{Optimizarea implementării}
Programul a fost implementat pe principiul de a fi funcțional: (aproape) toate
funcțiile și metodele întorc obiecte noi, în loc să le modifice pe cele
existente.

Acest lucru face implementarea mai clară și mai ușor de urmărit, însă aduce
penalizări destul de mare la timpul de rulare, întrucât limbajul C++ nu este
gândit pentru a fi utilizat într-un mod funcțional.
De asemenea, mulți algoritmi implementați în program au complexitatea $O$
suboptimă, deoarece au fost implementați în cel mai simplu mod, nu neapărat cel
mai rapid.
