\chapter{Detaliile tehnice ale limbajului Java}

În acest capitol vom detalia limbajul Java și modul de
funcționare a acestuia.

\section{Limbajul Java}

Java este un limbaj de programare orientat pe obiecte. Acesta a fost
dezvoltat de către Sun Microsystems (acum Oracle), iar prima versiune a
apărut în anul 1995.

Java s-a bazat pe sintaxa limbajului C, și a introdus noțiunea de
``scrie o data, rulează peste tot'' (Eng. ``write once, run
everywhere''). Spre deosebire de C și de C++, care trebuiesc compilate
pentru fiecare platformă țintă, Java a avut avantajul că trebuie
compilat o singura data, și va merge garantat pe toate platformele
suportate de limbaj.

\section{Java Bytecode}

Soluția limbajului Java pentru a fi independent de platformă este de a
transforma codul într-o reprezentare intermediara, în loc de direct în
cod binar pentru o anumită arhitectură .

Compilatorul Java (\texttt{javac}), transformă codul Java într-un limbaj
intermediar, numit Java Bytecode.

Acest limbaj este un limbaj de nivel scăzut (Eng. "low-level"), destinat în mod exclusiv
procesării de către mașini, spre deosebire de codul Java, care este
destinat oamenilor.

După ce compilatorul a procesat codul Java, provenit din fișiere .java în
format text, acesta salvează rezultatul în fișiere de tip clasă (.class)
în format binar.

\section{JVM}

Odată generate fișierele binare, acestea sunt executate pe o mașina
virtuală specifica limbajului Java --- numită \texttt{JVM}
(Eng. Java Virtual Machine).

Această mașina virtuală are rolul de a citi fișierele de clasă binare și
de a le interpreta.

Mașina virtuală este implementată ca o ``mașină cu stivă'' (Eng. Stack
machine), unde toate instrucțiunile limbajului bytecode interacționează
cu datele de pe o stivă controlată de aplicație.

Mașina virtuală însuși este implementată în C/C++, și este compilată în
cod binar direct, dependent de arhitectura calculatorului. Dezvoltatorii limbajului
Java sunt responsabili pentru corectitudinea și siguranța mașinii
virtuale, în timp ce dezvoltatorii de aplicații Java au garanția că dacă
codul lor Java este corect, atunci acesta va rula la fel, deterministic,
pe orice platforma.

În acestă privință, limbajul Java poate fi văzut ca un limbaj interpretat.
Comparând cu alte limbaje populare interpretate, ca de exemplu Python,
Ruby, sau Perl, ne-am aștepta ca și Java să fie la fel de încet ca
acestea~\cite{language_benchmarks}.
Totuși, Java obține performanțe mult mai bune decât
acestea. Acest fapt se datorează compilării tocmai-la-timp (Eng.\
just-in-time), în care atunci când interpretorul observă o secventa de
cod care este interpretată repetitiv de foarte multe ori, va genera
direct cod binari pentru aceasta.


\section{Fișierele clasa}

Fișierele de clasa Java sunt formate din 10
secțiuni\cite{classfile_sections}:

\begin{enumerate}
	\item
	      Constanta magică.
	\item
	      Versiunea fișierului.
	\item
	      Constantele clasei.
	\item
	      Permisiunile de acces.
	\item
	      Numele clasei din fișier.
	\item
	      Numele super clasei.
	\item
	      Interfețele pe care clasa le implementează.
	\item
	      Câmpurile clasei.
	\item
	      Metodele clasei.
	\item
	      Atribute ale clasei.
\end{enumerate}

În continuare voi da o scurta descriere a formatului secțiunilor.

\section{Secțiunile fișierelor clasa}

\subsection{Magic}

Toate fișierele clasă trebuiesc să înceapă cu un număr denumit constanta
magică. Acesta este folosit pentru a identifica în mod unic că acestea
sunt într-adevăr fișiere clasă. Numărul magic are o valoare memorabilă:
reprezentarea hexadecimală este \texttt{0xCAFEBABE},

\subsection{Versiunea}

Versiunea unui fișier clasă este dată de doua valori, versiunea majoră
\texttt{M} și versiunea minoră \texttt{m}. Versiunea clasei este atunci
reprezentata ca \texttt{M.m}. (e.g., \texttt{45.1}). Aceasta este
folosită pentru a menține compatibilitatea în cazul modificărilor
mașinii virtuale care interpretează clasa sau ale compilatorului care o
generează.

\subsection{Constantele clasei}

Tabela de constante este locul unde sunt stocate valorile literale
constante ale clasei:

\begin{itemize}
	\item Numere întregi.
	\item Numere cu virgula mobilă.
	\item Șiruri de caractere, care pot reprezenta la rândul lor:
	      \begin{itemize}
		      \item Nume de clase.
		      \item Nume de metode.
		      \item Tipuri ale metodelor.
	      \end{itemize}
	\item Informații compuse din datele anterioare:
	      \begin{itemize}
		      \item Referința către o metodă a unei clase.
		      \item Referința către o constantă a unei clase.
	      \end{itemize}
\end{itemize}

Toate celelalte tipuri de date compuse, cum ar fi metodele sau
câmpurile, vor conține indecși în tabela de constante.

Aproape toate tipurile de constante ocupă o singură poziție în tabelă, însă, din
motive istorice, unele constante ocupă două poziții.
Tot din motive istorice, tabela este indexată de la 1, și nu
de la 0, cum sunt celelalte.

\subsection{Permisiunile de acces}

Aceste permisiuni constau într-o mască de biți, care reprezintă
operațiile permise pe această clasă:

\begin{itemize}
	\item dacă clasa este publică, și poate fi accesată din afara pachetului acesteia.
	\item dacă clasa este finala, și dacă poate fi extinsă.
	\item dacă invocarea metodelor din super clasă să fie tratata special.
	\item dacă este de fapt o interfață, și nu o clasă.
	\item dacă este o clasă abstractă și nu poate fi instanțiată.
\end{itemize}

\subsection{Clasa curentă}

Reprezintă un indice în tabela de constante, unde sunt stocate
informații despre clasa curentă.

\subsection{Clasa super}

Reprezintă un indice în tabela de constante, cu informații despre clasa
din care a moștenit clasa curentă. Dacă este 0, clasa
curentă nu moștenește nimic: singura clasă fără o super clasă este
Object.

E.g. Pentru

\begin{lstlisting}[language=Java]
public class MyClass extends S implements I
\end{lstlisting}

Indicele corespunde lui \texttt{S}.

\subsection{Interfețele}

Reprezintă o colecție de indici în tabela de constante. Fiecare valoare
de la acei indici constituie o interfață implementată în mod direct de
către clasa curentă. Interfețele apar în ordinea declarată în fișierele
Java.

E.g. Pentru

\begin{lstlisting}[language=Java]
class MyClass extends S implements I1, I2
\end{lstlisting}

Primul indice ar corespunde lui \texttt{I1}, iar al doilea lui
\texttt{I2}.

\subsection{Câmpurile}\label{campurile}

Reprezintă informații despre câmpurile (Eng. Fields) clasei:
\begin{itemize}
	\item Permisiunile de acces: dacă este public sau privat, etc.
	\item Numele câmpului.
	\item Tipul câmpului.
	\item Alte atribute: dacă este depreciat, dacă are o valoare constantă, etc.
\end{itemize}

\subsection{Metodele}\label{metodele}

Reprezintă informații despre toate metodele clasei, inclusiv
constructorii:

\begin{itemize}
	\item Permisiuni de acces: dacă este publică sau privată,
        dacă este finală , dacă este abstractă.
	\item Numele metodei.
	\item Tipul metodei.
	\item În caz că nu este abstractă, codul metodei.
	\item Alte atribute:
	      \begin{itemize}
		      \item Ce excepții poate arunca.
		      \item Dacă este depreciată.
	      \end{itemize}
\end{itemize}

Codul metodei este partea cea mai importantă, iar formatul acestuia
urmează să fie detaliat ulterior.

\subsection{Atributele}

Reprezintă alte informații despre clasă, cum ar fi:
\begin{itemize}
	\item Clasele definite în interiorul acesteia.
	\item În caz că este o clasă anonimă sau definită local, metoda în care este definită.
	\item Numele fișierul sursă din care a fost compilată clasa.
\end{itemize}

\section{Tipuri de date}

În continuare, voi descrie din punct de vedere tehnic tipurile de date
întâlnite în fișierele de clasă.

\subsection{Tipuri de baza}

În formatul fișierelor clasa există trei tipuri elementare, toate bazate pe
întregi. În caz că un întreg are mai mulți octeți, aceștia au ordinea de
\texttt{big-endian}: cel mai semnificativ octet va fi mereu primul în
memorie.

\begin{longtable}[]{@{}ccc@{}}
	\toprule
	Nume        & Semantica                       & Echivalentul în C\tabularnewline
	\midrule
	\endhead
	\texttt{u1} & întreg pe un octet, fără semn   & \texttt{unsigned\ char}
	sau \texttt{uint8\_t}\tabularnewline
	\texttt{u2} & întreg pe doi octeți, fără semn & \texttt{unsigned\ short}
	sau \texttt{uint16\_t}\tabularnewline
	\texttt{u4} & întreg pe un octet, fără semn   & \texttt{unsigned\ int} sau
	\texttt{uint32\_t}\tabularnewline
	\bottomrule
\end{longtable}

În codul sursă al proiectului, acestea sunt tratate astfel:

\begin{lstlisting}[language=C++]
using u1 = uint8_t;
using u2 = uint16_t;
using u4 = uint32_t;
\end{lstlisting}

\subsection{Tipuri compuse}

\subsection{Constantele}

Fiecare constantă din tabela de constante începe cu o etichetă de 1
octet, care reprezintă datele și tipul structurii. Conținutul acesteia
variază în funcție de etichetă, însa indiferent de etichetă, conținutul
trebuie să aibă cel puțin 2 octeți.

\subsubsection{CONSTANT\_Class}

Corespunde valorii etichetei de 7 și conține un indice spre un alt câmp
în tabela de constante, de tipul \texttt{CONSTANT\_Utf8} --- un șir de
caractere. Acel șir de caractere va conține numele clasei.

\subsubsection{CONSTANT\_Fieldref}

Corespunde valorii etichetei de 9 și conține o referință spre câmpul
unei clase. Referința constă în doi indici, amândoi care arată spre
tabela de constante. Primul indice arată spre o constantă
\texttt{CONSTANT\_Class}, care reprezintă clasa sau interfața căreia
aparține metoda. Al doilea indice arată spre o constantă
\texttt{CONSTANT\_NameAndType}, care conține informații despre numele și
tipul câmpului.

\subsubsection{CONSTANT\_Methodref}

Corespunde valorii etichetei de 10 și conține o referința spre metoda
unei clase. Are o structură identică cu \texttt{CONSTANT\_Fieldref},
doar că primul indice arată neapărat spre o clasă, în timp ce al doilea
indice arată spre numele și tipul metodei.

\subsubsection{CONSTANT\_InterfaceMethodref}

Corespunde valorii etichetei de 11 și conține o referință spre metoda
unei interfețe. Are o structura identică cu
\texttt{CONSTANT\_Methodref}, doar că primul indice arată spre o
interfață.

\subsubsection{CONSTANT\_String}

Corespunde valorii etichetei de 8 și reprezintă un șir de caractere.
Conține un indice către o structură de tipul \texttt{CONSTANT\_Utf8}.

\subsubsection{CONSTANT\_Integer}

Corespunde valorii etichetei de 3 și conține un întreg pe 4 octeți.

\subsubsection{CONSTANT\_Float}

Corespunde valorii etichetei de 4 și conține un număr cu virgulă mobilă
pe 4 octeți.

\subsubsection{CONSTANT\_Long}

Corespunde valorii etichetei de 5 și conține un întreg pe 8 octeți. Din
motive istorice, ocupă 2 spatii în tabela de constante.

\subsubsection{CONSTANT\_Double}

Corespunde valorii etichetei de 6 și conține un număr cu virgulă mobilă
pe 8 octeți. Din motive istorice, ocupă 2 spatii în tabela de constante.

\subsubsection{CONSTANT\_NameAndType}

Corespunde valorii etichetei de 12. Descrie numele și tipul unui câmp
sau al unei metode, fără informații despre clasă. Conține doi indici,
amândoi către structuri de tipul \texttt{CONSTANT\_Utf8}. Primul
reprezintă numele, iar al doilea tipul.

\subsubsection{CONSTANT\_Utf8}

Corespunde valorii etichetei de 1. Reprezintă un șir de caractere
encodat în formatul UTF-8. Conține un întreg \texttt{length}, de tipul
\texttt{u2}, și apoi \texttt{length} octeți care descriu șirul în sine.
Din cauza că este encodat ca UTF-8, un singur caracter poate fi format
din mai multi octeți.

\subsubsection{CONSTANT\_MethodHandle}

Corespunde valorii etichetei de 15 și conține o referințe către un câmp,
o metodă de clasă, sau o metodă de interfață.

\subsubsection{CONSTANT\_MethodType}

Corespunde valorii etichetei de 16 și conține un indice către o
constantă \texttt{CONSTANT\_Utf8}, ce reprezintă tipul unei metode.

\subsubsection{CONSTANT\_InvokeDynamic}

Corespunde valorii etichetei de 18 și este folosit de către \texttt{JVM}
pentru a invoca o metodă polimorfică.

În cod \texttt{C++}, am reprezentat \texttt{cp\_info} astfel:

\begin{lstlisting}[language=C++]
struct cp_info {
    enum class Tag : u1 {
        CONSTANT_Class = 7,
        CONSTANT_Fieldref = 9,
        CONSTANT_Methodref = 10,
        CONSTANT_InterfaceMethodref = 11,
        CONSTANT_String = 8,
        CONSTANT_Integer = 3,
        CONSTANT_Float = 4,
        CONSTANT_Long = 5,
        CONSTANT_Double = 6,
        CONSTANT_NameAndType = 12,
        CONSTANT_Utf8_info = 1,
        CONSTANT_MethodHandle = 15,
        CONSTANT_MethodType = 16,
        CONSTANT_InvokeDynamic = 18,
    };

    Tag tag;
    std::vector<u1> data;
};
\end{lstlisting}

Iar structurile folosite pentru obiectivul propus au fost reprezentate
astfel:

\begin{lstlisting}[language=C++]
struct CONSTANT_Methodref_info {
    cp_info::Tag tag;
    u2 class_index;
    u2 name_and_type_index;
};
struct CONSTANT_Class_info {
    cp_info::Tag tag;
    u2 name_index;
};
struct CONSTANT_NameAndType_info {
    cp_info::Tag tag;
    u2 name_index;
    u2 descriptor_index;
};
\end{lstlisting}

\paragraph{\texorpdfstring{\texttt{field\_info}}{field\_info}}\label{field_info}

Fiecare câmp din cadrul unei clase este reprezentat printr-o structură
de tipul \texttt{field\_info}.

În cod \texttt{C++}, această structură a fost reprezentată astfel:

\begin{lstlisting}[language=C++]
struct field_info {
    u2 access_flags;
    u2 name_index;
    u2 descriptor_index;
    u2 attributes_count;
    std::vector<attribute_info> attributes;
};
\end{lstlisting}

Unde:
\begin{itemize}
	\item \texttt{name\_index} este o intrare în tabela de constante unde se afla o constantă de tipul \texttt{CONSTANT\_Utf8}.
	\item \texttt{descriptor\_index} arată spre o constantă de tipul \texttt{CONSTANT\_Utf8} și reprezintă tipul câmpului.
\end{itemize}

\paragraph{\texorpdfstring{\texttt{method\_info}}{method\_info}}\label{method_info}

Fiecare metodă a unei clase/interfețe este descrisă prin această
structură.

În cod \texttt{C++}, am implementat-o astfel:

\begin{lstlisting}[language=C++]
struct method_info {
    u2 access_flags;
    u2 name_index;
    u2 descriptor_index;
    u2 attributes_count;
    std::vector<attribute_info> attributes;
};
\end{lstlisting}

Unde \texttt{name\_index} și \texttt{descriptor\_index} au aceeași
interpretare ca și la \texttt{field\_info}.

Dacă metoda nu este abstractă, atunci în vectorul \texttt{attributes} se
va găsi un atribut de tipul \texttt{Code}, care conține bytecode-ul
corespunzător acestei metode.

\paragraph{\texorpdfstring{\texttt{attribute\_info}}{attribute\_info}}\label{attribute_info}

În \texttt{C++}, a fost implementată astfel:

\begin{lstlisting}[language=C++]
struct attribute_info {
    u2 attribute_name_index;
    u4 attribute_length;
    std::vector<u1> info;
};
\end{lstlisting}

Numele atributului determină modul în care octeții din vectorul
\texttt{info} sunt interpretați.

\subsubsection{Atributul de cod}

Pentru intențiile noastre, atributul de interes este cel de cod:

\subparagraph{\texorpdfstring{\texttt{Code\_attribute}}{Code\_attribute}}\label{code_attribute}

\begin{lstlisting}[language=C++]
struct Code_attribute {
    u2 attribute_name_index;
    u4 attribute_length;

    u2 max_stack;
    u2 max_locals;

    u4 code_length;
    std::vector<u1> code;

    u2 exception_table_length;
    struct exception {
        u2 start_pc;
        u2 end_pc;
        u2 handler_pc;
        u2 catch_type;
    };
    std::vector<exception> exception_table; // of length exception_table_length.
    u2 attributes_count;
    std::vector<attribute_info> attributes; // of length attributes_count.
};
\end{lstlisting}

Această structură este esențială pentru implementarea optimizării,
întrucât ea ne permite să determinăm metodele apelate din cadrul unei secvente de cod.
În continuare, o voi descrie detaliat:

\begin{itemize}
	\tightlist
	\item
	      \texttt{max\_stack}: Reprezintă adâncimea maximă a stivei mașinii
	      virtuale când această bucată de cod este interpretată.
	\item
	      \texttt{max\_locals}: Reprezintă numărul maxim de variabile locale
	      alocate în același timp când această bucată de cod este interpretată.
	\item
	      \texttt{code}: Codul metodei.
	\item
	      \texttt{exception\_table}: Excepțiile pe care le poate arunca metoda.
\end{itemize}

\subsubsection{Code}

Vectorul \texttt{code} din cadrul atributului \texttt{Code} reprezintă
bytecode-ul propriu-zis al metodei.

Acest vector conține instrucțiunile care sunt executate de către mașina
virtuală.

JVM-ul rulează ca o mașina cu stivă, iar toate instrucțiunile operează
pe această stivă. Rezultatul rulării unei instrucțiuni este modificarea
stivei: scoaterea și adăugarea de elemente în vraful acesteia.

Instrucțiunile au în general formatul~\cite{instruction_format}

\begin{verbatim}
nume_instr
operand1
operand2
...
\end{verbatim}

Cu un număr variabil de operanzi, prezenți în mod explicit în vectorul
de \texttt{cod}.

Fiecărei instrucțiuni îi corespunde un octet, denumit opcode. Fiecare
operand este fie cunoscut la compilare, fie calculat în mod dinamic la
rulare.

Cele mai multe operații nu au niciun operand dat în mod explicit la
nivelul instrucțiunii --- ele lucrează doar cu valorile din vârful stivei
la momentul execuției codului.

De exemplu:

Instrucțiunea \texttt{imul} are octetul 104 sau 0x68.
Aceasta extrage două valori din vârful stivei: \texttt{value1} și
\texttt{value2}. Ambele valori trebuie să fie de tipul \texttt{int}.
Rezultatul este înmulțirea celor două valori:
\texttt{result\ =\ value1\ *\ value2}, și este pus în vârful stivei.

Dintre cele peste o sută de instrucțiuni, noi suntem preocupați doar de
5 dintre acestea: cele care au de a face cu invocarea unei metode.

\textbf{invokedynamic}

Format:
\begin{verbatim}
invokedynamic
index1
index2
0
0
\end{verbatim}

Opcode-ul corespunzător acestei instrucțiuni este 186 sau
0xba.

Index1 și index2 sunt doi octeți. Aceștia sunt compuși în
\begin{lstlisting}
index = (index1 << 8) | index2
\end{lstlisting}
Unde \texttt{<<} reprezintă shiftare pe biți, iar \texttt{|}
reprezintă operația de sau pe biți.

Indicele compus reprezintă o intrare în tabela de constante. La locația
respectivă trebuie să se afle o structură de tipul
\texttt{CONSTANT\_MethodHandle}

\textbf{invokeinterface}

Format:
\begin{verbatim}
invokeinterface
index1
index2
count
0
\end{verbatim}

Opcode-ul corespunzător este 185 sau 0xb9.
\texttt{index1} și \texttt{index2} sunt folosiți, în mod similar ca la
\texttt{invokedynamic}, pentru a construi un indice în tabela
de constante.

La poziția respectivă în tabelă, trebuie să se găsească o structură de
tipul \texttt{CONSTANT\_Methodref}.

\texttt{count} trebuie să fie un octet fără semn diferit de 0. Acest
operand descrie numărul argumentelor metodei, și este necesar din motive
istorice (aceasta informație poate fi dedusa din tipul metodei).

\textbf{invokespecial}

Format:
\begin{verbatim}
invokespecial
index1
index2
\end{verbatim}

Opcode-ul corespunzător este 183 sau 0xb7. La fel ca la
\texttt{invokeinterface}, instrucțiunea conține un indice în tabela de constante
către o structură \texttt{CONSTANT\_Methodref}.

Această instrucțiune este folosită pentru a invoca constructorii
claselor.

\textbf{invokestatic}

Format:
\begin{verbatim}
invokestatic
index1
index1
\end{verbatim}

Opcode-ul corespunzător este 184 sau 0xb8.
Instrucțiunea invocă o metodă statică a unei clase.

La fel ca la \texttt{invokeinterface}, este construit un indice compus,
și folosit pentru a indexa tabela de constante.

\textbf{invokevirtual}

Format:
\begin{verbatim}
invokevirtual
index1
index1
\end{verbatim}

Opcode-ul corespunzător este 182 sau 0xb6, iar
interpretarea este la fel ca la \texttt{invokeinterface}.
Aceasta este cea mai comună
instrucțiune de invocare de funcții.

În \texttt{C++}, am reprezentat aceste instrucțiuni de interes astfel:

\begin{lstlisting}[language=C++]
enum class Instr {
    invokedynamic = 0xba,
    invokeinterface = 0xb9,
    invokespecial = 0xb7,
    invokestatic = 0xb8,
    invokevirtual = 0xb6,
};
\end{lstlisting}

\subsection{ClassFile}\label{classfile}

Folosind definițiile anterioare, putem descrie un fișier de clasă binar
în C++:

\begin{lstlisting}[language=C++]
struct ClassFile {
    u4 magic; // Should be 0xCAFEBABE.

    u2 minor_version;
    u2 major_version;

    u2 constant_pool_count;
    std::vector<cp_info> constant_pool;

    u2 access_flags;

    u2 this_class;
    u2 super_class;

    u2 interface_count;
    std::vector<interface_info> interfaces;

    u2 field_count;
    std::vector<field_info> fields;

    u2 method_count;
    std::vector<method_info> methods;

    u2 attribute_count;
    std::vector<attribute_info> attributes;
};
\end{lstlisting}

Pentru a vedea un fișier clasa analizat în detaliu, va puteți
uita la appendix-ul Studiu de Caz \ref{appendix:studiu_de_caz}.
