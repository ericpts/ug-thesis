\chapter{Optimizari de dimensiune}\label{capitolul_optimizari_de_dimensiune}

\section{Eliminarea functiilor nefolosite}

Aceasta lucrare contine o abordare de optimizare a dimensiunii
programelor bazata pe eliminarea functiilor nefolosite.

Desi metrica principala este aceea de dimensiune, reducerea
numarului de instructiuni ale programului poate avea si beneficii
asupra vitezei de executie, din cauza interactiunilor cu cache-ul
calculatoarelor. Pe de alta parte, acest beneficiu este destul de
minor, si nu preprezinta principala motivatie.

In continuare, voi descrie conceptele necesare pentru a intelege
cum putem implementa acest fel de optimizare.

\section{Functie/Metoda}

In domeniul limbajelor de programare, o functie \s{F} este
formata dintr-o colectie de instructiuni tratate ca un intreg,
impreuna cu un protocol pentru executarea acelor instructiuni.

O metoda \s{M} reprezinta o functie asociata unei clase.

In limbajul Java este imposibil sa avem o functie care sa nu
apartina unei metode.
Desi definitiile nu sunt echivalente, deoarece in Java nu exista
functii propriu-zise, ci doar metode, termenii de functie si de metoda
sunt considerati interschimbabili.

Protocolul de executare a functiilor variaza de la limbaj la
limbaj si nu este important pentru scopurile noastre.

De exemplu, functia f
\begin{lstlisting}[language=Python, numbers=left, firstnumber=0]
def f(a, b, c):
    a = a + b
    return c
\end{lstlisting}
contine 2 instructiuni, corespunzatoare liniilor 1-2:

$(i_1)\ a = a + b$

$(i_2)\ return\ c$

Desi in acest exemplu functia \s{F} incepe de pe pozitia 0,
intr-un program format din mai multe functii locul de inceput al
functiei variaza (indicele primei instructiuni).

Vom spune ca doua instructiuni sunt egale daca se afla pe aceeasi
linie. i.e., reprezinta aceeasi pozitie in program.

Pentru programul format din
\begin{lstlisting}[language=Python, numbers=left, firstnumber=0]
def g(a, b, c):
    a = a + b
    a = a * b
    return a
def f(a, b, c):
    a = a + b
    return c
\end{lstlisting}
instructiunile $i_1$ si $i_5$ \textbf{nu} sunt egale.

Vom spune ca o instructiune $\mathbf{i}$ apartine functiei
\s{F} ddaca exista o linie a lui \s{F} unde putem gasi
instructiunea respectiva.

\section{Punctul de intrare principal}

Fie \s{P} un program, executat in ordine secventiala.
Punctul principal de intarare al lui \s{P} reprezinta locul de
unde incepe executia programului -- prima instructiune executata.

Acest loc va fi notat cu \texttt{main}.

\section{Sirul de executie al unui program}

Un sir de executie \s{S} al programului \s{P} reprezinta o inaltuire
finita de instructiuni executate secvential, incepand cu
\texttt{main} si terminandu-se cu ultima instructiune executata de
catre \s{P} inainte ca programul sa se termine.
Pentru ca un sir de executie \s{S} sa fie valid, trebuie sa
existe un input pe care, daca rulam programul, acesta sa
execute fix instructiunile lui \s{S}.

\section{Functii nefolosite}

O functie a unei clase poate fi eliminata daca nu exista niciun
sir de executie valid al programului care sa treaca prin acea
functie.

Vom nota cu $elim(\mathcal{F}) = 1$ daca functia \s{F} poate fi
eliminata deoarece este nefolosita.

Cu alte cuvinte, $elim(\mathcal{F}) = 1$ ddaca

\[
	\text{Pentru oricare } \mathcal{S} \text{ sir de executie, nu
		exista } \mathbf{i} \text{ din } \mathcal{S} \text { care sa faca
		parte din corpul functiei } \mathcal{F}.
\]

\subsection{Sirurile de executie - in practica}

Problema determinarii tuturor sirurilor de executie ale unui
program este o problema grea, intrucat aceasta ar implica rularea
programului pe fiecare input posibil.

Din cauza ca unele programe pot sa fie definite doar partial
(comportamentul lor sa fie nedefinit pe anumite clase de input),
aceasta problema poate fi echivalenta cu Problema Opririi (eng. "The
Halting Problem") \cite{the_halting_problem}: nu avem cum sa stim
daca un input al programului este bun sau nu fara sa rulam
programul, insa nu putem determina daca un program se va termina
vreodata.

Deoarece problema opririi este intractabila, determinarea tuturor
sirurilor de executie posibile ale lui program arbitrar este de
asemenea o problema intractabila.

\subsection{Supersirul de executie}\label{supersirul_de_executie}

Aceasta lucrare va construi o aproximare -- un "supersir" de executie format
dintr-o superpozitie a tuturor sirurilor de executie existente, inclusiv pe cele
invalide (pentru care nu exista un input care sa le genereze).

De exemplu, pentru programul
\begin{lstlisting}[language=Python, numbers=left]
def main():
    if False:
        f()

def f():
    if True:
        return
    g()

def g():
    pass

def h():
    pass
\end{lstlisting}

sirul de executie folosit in problema noastra va contine si
secventa corespunzatoare lantului de apeluri \(main() -> f()\),
cat si lantului \(main() -> f() -> g()\),
chiar daca acestea sunt invalide.

Totodata, sirul de executie generat \textbf{nu} va contine
secventa \(main() -> h()\), sau orice secventa care sa il contina
pe h, din cauza ca aceste secvente nu sunt siruri de executie
(nu exista nicio secventa de instructiuni secventiale
care sa porneasca din functia \texttt{main} si sa ajunga in
functia h).

\subsection{Graful apelurilor de functii}\label{graful_apelurilor}

Vom construi un graf \s{G} al metodelor: fiecarui nod ii va
corespunde o metoda prezenta in program, iar fiecare metoda va
avea un singur nod corespunzator.

Supersirul de executie considera toate secventele de instructiuni
dintr-o metoda, chair daca acestea sunt invalide.
Prin urmare, toate posibilele cai de apel vor fi considerate de
catre acesta.

Asadar, putem reduce problema constructiei supersirului la
problema construirii grafului \s{G}.

In acest graf, exista o muchie de la metoda F la metoda
G daca in secventa de instructiuni a lui F (corpul functiei)
exista un apel catre functia G.

Avand acest graf generat, putem computa multimea \s{M} a metodelor
care sunt accesibile pornind din nodul corespunzator functiei care
contine punctului principal de intrare.
Pentru a optimiza programul, vom elimina toate metodele care nu
apartin acestei multimi.

Demonstratie ca acest procedeu este corect:
\begin{lemma}

	Fie m o metoda care nu apartine lui \s{M}.

	Acest fapt inseamna ca in supersirul de executie nu exista nicio
	instructiune care sa apartina lui m.

	Cum supersirul de executie include toate sirurile de executie
	valide, inseamna ca nu exista niciun sir de executie valid care,
	la un moment dat, sa apeleze functia m.

	Prin urmare, programul obtinut prin eliminarea functiei m va fi
	echivalent cu programul initial.
\end{lemma}

\subsection{Algoritmul pentru eliminarea
functiilor}\label{algoritm_naiv_pentru_eliminare}

Pentru a construi graful, trebuie sa putem deduce pentru fiecare
metoda care sunt metodele pe care aceasta le apeleaza.
Acest lucru este dependent de limbajul asupra caruia se aplica
optimizarea, asadar il vom considera deja implementat.

Putem considera astfel functia
\begin{lstlisting}[language=Python]
def direct_calees(m: Method) -> [Method]
    return all methods directly called by m.
\end{lstlisting}
ca fiind la dispozitia noastra.

Pentru a forma multimea metodelor accesibile din \texttt{main},
putem utiliza o parcugere in latime a grafului.

\begin{lstlisting}[language=Python]
def reachable_methods(main: Method) -> [Method]:
    coada = [main]
    at = 0
    while at < size(coada):
        m = coada[at]
        at = at + 1
        for next in direct_calees(m):
            if next not in coada:
                coada.push(next)
    return coada
\end{lstlisting}

Avand multimea generata, ramane doar sa eliminam functiile
care nu fac partea din ea:

\begin{lstlisting}[language=Python]
def optimize_for_size(p: Program) -> Program:
    main = p.main_method()
    used_methods = reachable_methods(main)
    for m in p.all_methods():
        if m not in used_methods:
            p.remove_method(m)
    return p
\end{lstlisting}
