\chapter{Optimizari de dimensiune}\label{capitolul_optimizari_de_dimensiune}

\section{Eliminarea functiilor nefolosite}

Această lucrare conține o abordare de optimizare a dimensiunii
programelor bazată pe eliminarea funcțiilor nefolosite.

Deși metrica principală este aceea de dimensiune, reducerea
numarului de instrucțiuni ale programului poate avea și beneficii
asupra vitezei de executie, din cauza interactiunilor cu cache-ul
calculatoarelor. Pe de altă parte, acest beneficiu este destul de
minor, și nu preprezintă principala motivatie.

In continuare, voi descrie conceptele necesare pentru a intelege
cum putem implementa acest fel de optimizare.

\section{Functie/Metoda}

In domeniul limbajelor de programare, o functie \s{F} este
formată dintr-o colecție de instrucțiuni tratate ca un întreg,
împreună cu un protocol pentru executarea acelor instructiuni.

O metoda \s{M} reprezintă o funcție asociată unei clase.

In limbajul Java este imposibil să avem o funcție care să nu
aparțină unei metode.
Deși definițiile nu sunt echivalente, deoarece în Java nu există
funcții propriu-zise, ci doar metode, termenii de funcție și de metodă
sunt considerați interschimbabili.

Protocolul de executare a funcțiilor variază de la limbaj la
limbaj și nu este important pentru scopurile noastre.

De exemplu, functia f
\begin{lstlisting}[language=Python, numbers=left, firstnumber=0]
def f(a, b, c):
    a = a + b
    return c
\end{lstlisting}
conține 2 instructiuni, corespunzatoare liniilor 1-2:

$(i_1)\ a = a + b$

$(i_2)\ return\ c$

Deși în acest exemplu funcția \s{F} incepe de pe pozitia 0,
într-un program format din mai multe funcții locul de inceput al
funcției variază (indicele primei instructiuni).

Vom spune că două instrucțiuni sunt egale dacă se află pe aceeași
linie. i.e., reprezintă aceeași poziție în program.

Pentru programul format din
\begin{lstlisting}[language=Python, numbers=left, firstnumber=0]
def g(a, b, c):
    a = a + b
    a = a * b
    return a
def f(a, b, c):
    a = a + b
    return c
\end{lstlisting}
instructiunile $i_1$ și $i_5$ \textbf{nu} sunt egale.

Vom spune ca o instrucțiune $\mathbf{i}$ aparține funcției
\s{F} dacă există o linie a lui \s{F} unde putem gasi
instructiunea respectivă.

\section{Punctul de intrare principal}

Fie \s{P} un program, executat în ordine secvențială.
Punctul principal de intrare al lui \s{P} reprezintă locul de
unde incepe execuția programului -- prima instrucțiune executată.

Acest loc va fi notat cu \texttt{main}.

\section{Sirul de executie al unui program}

Un șir de execuție \s{S} al programului \s{P} reprezintă o înlănțuire
finită de instrucțiuni executate secvențial, incepând cu
\texttt{main} si terminându-se cu ultima instrucțiune executată de
către \s{P} înainte ca programul să se termine.
Pentru ca un șir de execuție \s{S} să fie valid, trebuie să
existe un input pe care, dacă rulam programul, acesta să
execute fix instrucțiunile lui \s{S}.

\section{Funcții nefolosite}

O funcție a unei clase poate fi eliminată dacă nu există niciun
șir de execuție valid al programului care să treacă prin acea
funcție.

Vom nota cu $elim(\mathcal{F}) = 1$ dacă funcția \s{F} poate fi
eliminată deoarece este nefolosită.

Cu alte cuvinte, $elim(\mathcal{F}) = 1$ dacă

\[
	\text{Pentru oricare } \mathcal{S} \text{ șir de execuție, nu
		există } \mathbf{i} \text{ din } \mathcal{S} \text { care să facă
		parte din corpul funcției } \mathcal{F}.
\]

\subsection{Sirurile de execuție - în practică}

Problema determinarii tuturor șirurilor de execuție ale unui
program este o problema grea, intrucât aceasta ar implica rularea
programului pe fiecare input posibil.

Din cauza că unele programe pot să fie definite doar parțial
(comportamentul lor sa fie nedefinit pe anumite clase de input),
această problemă poate fi echivalentă cu Problema Opririi (eng. "The
Halting Problem") \cite{the_halting_problem}: nu avem cum să știm
dacă un input al programului este bun sau nu fară să rulam
programul pe acel input. Totodată, nu putem determina dacă un program se va termina
vreodată.

Deoarece problema opririi este intractabilă, determinarea tuturor
șirurilor de execuție posibile ale lui program arbitrar este de
asemenea o problema intractabilă.

\subsection{Superșirul de execuție}\label{supersirul_de_executie}

Aceasăa lucrare va construi o aproximare -- un "superșir" de execuție format
dintr-o superpoziție a tuturor șirurilor de execuție existente, inclusiv pe cele
invalide (pentru care nu există un input care să le genereze).

De exemplu, pentru programul
\begin{lstlisting}[language=Python, numbers=left]
def main():
    if False:
        f()

def f():
    if True:
        return
    g()

def g():
    pass

def h():
    pass
\end{lstlisting}

șirul de execuție folosit în problema noastră va conține și
secvența corespunzătoare lanțului de apeluri \(main() -> f()\),
cât și lanțului \(main() -> f() -> g()\),
chiar dacă acestea sunt invalide.

Totodata, șirul de execuție generat \textbf{nu} va conține
secvența \(main() -> h()\), sau orice secvență care să îl conțină
pe h, din cauză că aceste secvențe nu sunt șiruri de execuție
(nu există nicio secvență de instrucțiuni secvențiale
care să pornească din funcția \texttt{main} și să ajungă în
functia h).

\subsection{Graful apelurilor de funcții}\label{graful_apelurilor}

Vom construi un graf \s{G} al metodelor: fiecărui nod îi va
corespunde o metodă prezentă în program, iar fiecare metodă va
avea un singur nod corespunzator.

Superșirul de execuție consideră toate secvențele de instrucțiuni
dintr-o metodă, chiar dacă acestea sunt invalide.
Așadar, superșirul va considera toate  posibilele căi de apel către alte funcții.

Așadar, putem reduce problema construcției superșirului la
problema construirii grafului \s{G}.

In acest graf, există o muchie de la metoda F la metoda
G dacă în secvența de instrucțiuni a lui F (corpul funcției)
există un apel către funcția G.

Având acest graf generat, putem computa mulțimea \s{M} a metodelor
care sunt accesibile pornind din nodul funcției care
conține punctul principal de intrare.
Pentru a optimiza programul, vom elimina toate metodele care nu
aparțin acestei mulțimi.

Demonstrație că acest procedeu este corect:
\begin{lemma}

	Fie m o metodă care nu aparține lui \s{M}.

	Acest fapt inseamnă că în superșirul de execuție nu există nicio
	instrucțiune care să aparțină lui m.

	Cum superșirul de execuție include toate șirurile de execuție
	valide, inseamnă că nu există niciun șir de execuție valid care,
	la un moment dat, sa apeleze functia m.

	Prin urmare, programul obținut prin eliminarea funcției m va fi
	echivalent cu programul initial.
\end{lemma}

\subsection{Algoritmul pentru eliminarea
functiilor}\label{algoritm_naiv_pentru_eliminare}

Pentru a construi graful, trebuie să putem deduce pentru fiecare
metodă care sunt metodele pe care aceasta le apelează.
Acest lucru este dependent de limbajul asupra caruia se aplică
optimizarea, așadar îl vom considera deja implementat.

Putem considera astfel functia
\begin{lstlisting}[language=Python]
def direct_calees(m: Method) -> [Method]
    return all methods directly called by m.
\end{lstlisting}
ca fiind la dispozitia noastră.

Pentru a forma mulțimea metodelor accesibile din \texttt{main},
putem utiliza o parcugere în lățime a grafului.

\begin{lstlisting}[language=Python]
def reachable_methods(main: Method) -> [Method]:
    coada = [main]
    at = 0
    while at < size(coada):
        m = coada[at]
        at = at + 1
        for next in direct_calees(m):
            if next not in coada:
                coada.push(next)
    return coada
\end{lstlisting}

Având mulțimea generată, ramâne doar să eliminăm funcțiile
care nu fac partea din ea:

\begin{lstlisting}[language=Python]
def optimize_for_size(p: Program) -> Program:
    main = p.main_method()
    used_methods = reachable_methods(main)
    for m in p.all_methods():
        if m not in used_methods:
            p.remove_method(m)
    return p
\end{lstlisting}
