\chapter{Introducere}

Eliminarea codului inutil (eng. "Dead code elimination")
\cite{wiki:deadcodeelimination} este o optimizare clasică.  Ea
presupune eliminarea dintr-un program a codului care nu afectează
rezultatul computației.

Codul poate să fie eliminat dacă, de exemplu, nu există niciun fir
de execuție care să conțină instrucțiunile respective, sau dacă
acel cod nu are efecte laterale.

Acest tip de optimizare este implementată tradițional în
limbajele compilate, precum C, cât și în cele `compilate-la-timp'
(eng. `just-in-time') precum Java sau JavaScript.  Beneficiile
principale aduse de către optimizare sunt:

\begin{enumerate}
	\item Reducerea dimensiunii programului
	\item Creșterea vitezei de execuție
\end{enumerate}

Cu cât un limbaj este mai dinamic, și permite mai multe schimbări ale
comportamentului la rulare, cu atât eliminarea de cod nefolosit
devine o sarcină mai grea.

De exemplu, echipa care dezvoltă motorul de JavaScript pentru
browser-ul Internet Explorer a avut probleme de corectitudine în
optimizările realizate, datorită naturii dinamice a limbajului
JavaScript \cite{deadcodeeliminationforbeginners}.

Deși pentru un programator nu este natural să creeze cod
nefolosit, acest lucru nu înseamnă ca principiul eliminării
acestuia nu poate fi aplicat; compilatoarele în sine, prin modul
lor de a trece de mai multe ori prin codul sursa și de a avea mai
multe reprezentări intermediare, pot genera cod nefolosit.

Limbajul C, prin includerea de fișiere mot-a-mot, este un exemplu
bun pentru acest lucru: un program de câteva linii, care afișează
"Hello, world!", ajunge sa aibă, înaintea eliminării codului mort,
câteva mii de linii și sute de funcții nefolosite. Acest lucru se datorează
nevoii includerii librăriei standard.

\section{Eliminarea metodelor}

Eliminarea metodelor și funcțiilor este un una dintre multele
posibilități de a scapă de cod nefolosit.

Predeul constă în îndepărtarea dintr-un program a funcțiilor și a
metodelor care nu sunt niciodată apelate.

Această lucrare va explora aceasta optimizare particulară, atât
teoretic, cât și implementat in limbajul Java.

\section{Structura lucrării}

În prima parte a acestei lucrări voi formaliza în mod teoretic
problema de a optimiza un program.
În a doua parte voi descrie limbajul Java, și modul de
funcționare a acestuia.
În a treia parte voi detalia cum putem adapta teoria de
optimizare pentru programe dezvoltate în limbajul Java.
În ultima parte voi prezenta detaliile de implementare ale
optimizatorului de Java.
