% Some commands used in this file
\newcommand{\package}{\emph}

\chapter{Introducere}

Eliminarea codului nefolosit (eng. "Dead code elimination")
\cite{wiki:deadcodeelimination} este o optimizare clasica.
Ea presupune eliminarea dintr-un program a codului care nu afecteaza rezultatul.
Acest lucru poate sa fie datorat faptului ca nu exista niciun fir de
executie care sa ajunga la instructiunile respective, sau ca acele instructiuni
nu au efecte laterale care sa efecteze restul programului.

Acest tip de optimizare este implementata traditional in limbajele compilate precum C, cat si
in cele `compilate-la-timp' (eng. `just-in-time') precum Java sau JavaScript.
Beneficiile principale aduse sunt:

\begin{enumerate}
    \item Reducerea dimensiunii programului
    \item Cresterea vitezei executiei
\end{enumerate}

Cu cat un limbaj este mai dinamic, si permite mai multe schimbari ale
comportamentului uzual, cu atat eliminarea de cod nefolosit devine mai grea.
De exemplu, echipa care dezvolta Internet Explorer a avut probleme de
corectitudine in optimizarile realizate, datorita naturii dinamice a limbajului
JavaScript \cite{deadcodeeliminationforbeginners}.

Desi pentru un programator nu este natural sa creeze cod nefolosit, acest lucru
nu inseamna ca principiul eliminarii acestuia este ineficient; compilatoarele in
sine, prin natura lor de a trece de mai multe ori prin codul sursa si de a avea
mai multe reprezentari intermediare pot genera cod nefolosit.

Limbajul C, prin natura sa de a include fisiere mot-a-mot, este o exemplificare
foarte buna pentru acest lucru: un program de cateva linii, care afiseaza
"Hello, world!" la ecran, ajunge sa aiba, inaintea eliminarea codului mort,
cateva mii de linii si sute de functii nefolosite. Acest lucru se datoreaza
nevoii includerii librariei standard.

Eliminarea metodelor si functiilor este un una dintre multele
posibilitati de a scapa de cod nefolosit.
In limbajul Java, din cauza natura acestuia care permita incarcarea si
executarea de cod arbitrar la rulare, eliminarea functiile intr-un mod general
corect este un procedeu imposibil - daca o functie pusa la dispozitie de o librarie
este eliminata, nu se poate garanta ca pe viitor clasa care contine functie nu
va fi incarcata dinamic intr-un mod neprevazut.

%% \section{Features}
%% \label{sec:features}
%%
%% The rest of this document shows off a few features of the template
%% files.  Look at the source code to see which macros we used!
%%
%% The template is divided into \TeX{} files as follows:
%% \begin{enumerate}
%% \item \texttt{thesis.tex} is the main file.
%% \item \texttt{extrapackages.tex} holds extra package includes.
%% \item \texttt{layoutsetup.tex} defines the style used in this document.
%% \item \texttt{theoremsetup.tex} declares the theorem-like environments.
%% \item \texttt{macrosetup.tex} defines extra macros that you may find
%%   useful.
%% \item \texttt{introduction.tex} contains this text.
%% \item \texttt{sections.tex} is a quick demo of each sectioning level
%%   available.
%% \item \texttt{refs.bib} is an example bibliography file.  You can use
%%   Bib\TeX{} to quote references.  For example, read
%%   \cite{bringhurst1996ets} if you can get a hold of it.
%% \end{enumerate}
%%
%%
%% \subsection{Extra package includes}
%%
%% The file \texttt{extrapackages.tex} lists some packages that usually
%% come in handy.  Simply have a look at the source code.  We have
%% added the following comments based on our experiences:
%% \begin{description}
%% \item[REC] This package is recommended.
%% \item[OPT] This package is optional.  It usually solves a specific
%%   problem in a clever way.
%% \item[ADV] This package is for the advanced user, but solves a problem
%%   frequent enough that we mention it. Consult the package's
%%   documentation.
%% \end{description}
%%
%% As a small example, here is a reference to the Section \emph{Features}
%% typeset with the recommended \package{varioref} package:
%% \begin{quote}
%%   See Section~\vref{sec:features}.
%% \end{quote}
%%
%%
%% \subsection{Layout setup}
%%
%% This defines the overall look of the document -- for example, it
%% changes the chapter and section heading appearance.  We consider this
%% a `do not touch' area.  Take a look at the excellent \emph{Memoir}
%% documentation before changing it.
%%
%% In fact, take a look at the excellent \emph{Memoir} documentation,
%% full stop.
%%
%%
%% \subsection{Theorem setup}
%%
%% This file defines a bunch of theorem-like environments.
%%
%% \begin{theorem}
%%   An example theorem.
%% \end{theorem}
%%
%% \begin{proof}
%%   Proof text goes here.
%% \end{proof}
%%
%% Note that the q.e.d.\ symbol moves to the correct place automatically
%% if you end the proof with an \texttt{enumerate} or
%% \texttt{displaymath}.  You do not need to use \verb-\qedhere- as with
%% \package{amsthm}.
%%
%% \begin{theorem}[Some Famous Guy]
%%   Another example theorem.
%% \end{theorem}
%%
%% \begin{proof}
%%   This proof
%%   \begin{enumerate}
%%   \item ends in an enumerate.
%%   \end{enumerate}
%% \end{proof}
%%
%% \begin{proposition}
%%   Note that all theorem-like environments are by default numbered on
%%   the same counter.
%% \end{proposition}
%%
%% \begin{proof}
%%   This proof ends in a display like so:
%%   \begin{displaymath}
%%     f(x) = x^2.
%%   \end{displaymath}
%% \end{proof}
%%
%%
%% \subsection{Macro setup}
%%
%% For now the macro setup only shows how to define some basic macros,
%% and how to use a neat feature of the \package{mathtools} package:
%% \begin{displaymath}
%%   \abs{a}, \quad \abs*{\frac{a}{b}}, \quad \abs[\big]{\frac{a}{b}}.
%% \end{displaymath}
