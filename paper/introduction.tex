\chapter{Introducere}

Eliminarea codului inutil (Eng. "Dead code elimination")
\cite{wiki:deadcodeelimination} este o optimizare clasică.  Ea
presupune eliminarea dintr-un program a codului care nu afectează
rezultatul computației.

Codul poate să fie eliminat dacă, de exemplu, nu există niciun fir
de execuție care să conțină instrucțiunile respective, sau dacă
acel cod nu are efecte laterale.

Acest tip de optimizare este implementată tradițional în
limbajele compilate, precum C, cât și în cele `compilate-la-timp'
(Eng. `just-in-time') precum Java sau JavaScript.  Beneficiile
principale aduse de către optimizare sunt:

\begin{enumerate}
	\item Reducerea dimensiunii programului
	\item Creșterea vitezei de execuție
\end{enumerate}

Cu cât un limbaj este mai dinamic, și permite mai multe schimbări ale
comportamentului la rulare, cu atât eliminarea de cod nefolosit
devine o sarcină mai grea.

De exemplu, echipa care dezvoltă motorul de JavaScript pentru
browser-ul Internet Explorer a avut probleme de corectitudine în
optimizările realizate, datorită naturii dinamice a limbajului
JavaScript \cite{deadcodeeliminationforbeginners}.

Deși pentru un programator nu este natural să creeze cod
nefolosit, acest lucru nu înseamnă ca principiul eliminării
acestuia nu poate fi aplicat; compilatoarele în sine, prin modul
lor de a trece de mai multe ori prin codul sursă și de a avea mai
multe reprezentări intermediare, pot genera cod nefolosit.

Limbajul C, prin includerea de fișiere mot-a-mot, este un exemplu
bun pentru acest lucru: un program de câteva linii, care afișează
"Hello, world!", ajunge să aibă, înaintea eliminării codului mort,
câteva mii de linii și sute de funcții nefolosite. Acest lucru se datorează
nevoii includerii bibliotecii standard.

\section{Eliminarea metodelor}

Eliminarea metodelor și funcțiilor este un una dintre multele
posibilități de a scăpa de cod nefolosit.

Procedeul constă în îndepărtarea dintr-un program a funcțiilor și a
metodelor care nu sunt niciodată apelate.

Această lucrare va explora această optimizare particulară, atât
teoretic, cât și implementat în limbajul Java.

\section{Muncă anterioară}

\subsection{C si C++}

Compilatorul gcc este capabil să elimine funcțiile nefolosite,
însă acest comportament nu este default.

Pentru exemplificare, vom considera programul
\begin{lstlisting}[language=C, title=main.c,
label=c:program]
#include <stdio.h>

void foo()
{
    puts("in foo!");
}

void bar()
{
    puts("in bar!");
}

int main()
{
    foo();
    return 0;
}
\end{lstlisting}

Dacă îl compilăm cu \textbf{gcc} folosind opțiunea \texttt{-Os}
(permite optimizările pentru dimensiune), atunci binarul generat
va conține atât funcția \texttt{foo}, care este apelată din
\texttt{main}, cât și funcția \texttt{bar}, care nu este
niciodată folosită.

\begin{lstlisting}[language=Bash]
$ gcc main.c -Os
$ nm a.out | grep "foo"
000000000000064a T foo
$ nm a.out | grep "bar"
0000000000000656 T bar
$
\end{lstlisting}

Acest lucru se datorează faptului că optimizatorul pentru C
lucrează cu simboluri, nu direct cu funcții.  Diferența este că
un simbol poate reprezenta atât o funcție, cât și o variabilă.
Așadar, când optimizatorul decidă dacă să elimine un simbol,
acesta nu consideră dacă acel simbol este o funcție sau un obiect
(o variabilă se mai numește și obiect).

În limbajul C, un obiect global trebuie să fie inițializat cu o
valoare constantă (i.e., poate fi evaluată la compilare), ceea ce
înseamnă că inițializările nu pot avea efecte laterale
\cite{c_static_init}.

Pe de altă parte, în limbajul C++, inițializările pot conține
expresii arbitrare, care necesită evaluarea acestora la timpul
rulării \cite{cpp_static_init} și care pot avea efecte laterale.

Optimizarea de a elimina funcțiile nefolosite nu este activată în
mod implicit deoarece optimizatorul folosit în gcc lucrează și
pentru C, dar și pentru C++.

Această optimizare ar altera comportamentul unui program C++ care
folosește efecte laterale la inițializare, întrucât dacă
optimizatorul elimină din program un simbol care corespunde unei
variabile, acea variabilă nu mai este inițializată, la rularea
programului, iar efectul lateral corespunzător inițializării ei
nu se mai realizează.

\subsubsection{Opțiuni speciale date compilatorului}

Optimizarea de a elimina simboluri nefolosite nu este aplicabilă
oricărui program. Totodată, dacă programatorul îi garantează
compilatorului că programul nu folosește inițializări cu efecte
laterale, compilatorul gcc este capabil de a elimina funcții
nefolosite.

Vom folosi în continuare programul definit la \ref{c:program}.
Conform \cite{c_enable_optimization}, este necesar să pasăm
opțiuni suplimentare compilatorului, iar acesta va elimina
funcția nefolosită:

\begin{lstlisting}[language=Bash]
$ gcc main.c -Wl,-static -Wl,--gc-sections -fdata-sections -ffunction-sections -Os
$ nm a.out | grep "foo"
0000000000400acd T foo
$ nm a.out | grep "bar"
$
\end{lstlisting}

În concluzie, pentru limbajele C și C++, este nevoie ca
programatorul să garanteze că aplicarea optimizării păstrează
corectitudinea programului.

\subsection{Java}

Compilatorul standard de Java, \texttt{javac}, este incapabil să
efectueaze optimizarea de a elimina metode nefolosite.

În limbajul C, compilatorul gcc știe că programul își începe
execuția din funcția \texttt{main}, și că acesta este singurul
mod în care programul poate fi folosit.

În limbajul Java, compilatorul \texttt{javac}, lucrează cu câte
un fișier o data: acesta nu are conceptul de proiect sau
executabil, ci doar de fișiere clasă.
Compilatorul nu cunoaște ce funcții pot fi apelate, sau
din ce locuri, deci nu poate efectua nicio optimizare.

\subsubsection{Android}

Deși codul Java arbitrar este imposibil de optimizat, structuri
particulare de proiecte permit eliminarea de cod nefolosit.

Acest lucru este folosit pentru aplicațiile dezvoltate pentru
sistemul de operare Android \cite{android_proguard}.
Programul ales pentru a realiza sarcina de optimizare este
\texttt{Proguard} \cite{proguard}.

Soluția acestui program pentru a rezolva problema pe care o
întămpină \texttt{javac} este să îi ceară programatorului să
specifice în mod explicit modurile în care programul este rulat
(e.g., de unde poate începe execuția).

Având această informație, \texttt{proguard} poate analiza static
proiectul pentru a deduce care metode pot fi eliminate.


\section{Asumpții}

Înainte de a începe descrierea problemei, voi expune asumpțiile pe care le-am
făcut în implementarea aplicației pentru Java.
Deși aceste asumpții par destul de restrictive, cele mai multe proiecte normale,
atât de Android, cât și normale, le vor satisface.

Programul dezvoltat în această lucrare optimizează programele Java la compilare.
În urmare, acesta nu poate să trateze invocarea de metode la rulare.

Programul va emite un mesaj de eroare în caz că detectează că asumpția este
încălcată,

\subsection{Reflecția}

Reflecția constă în introspecția programului la rulare -- spre exemplu,
identificarea câmpurilor sau metodelor unei clase.

Deși reflecția nu este de obicei folosită în acest scop (decât în circumstanțe
specifice, de exemplu implementarea altor limbaje), aceasta permite programului
să apeleze metode care nu sunt cunoscute decât la timpul rulării.

De exemplu, programul \ref{java_reflection}
\begin{lstlisting}[language=Java]
public class Main {
  public int add(int a, int b)
  {
     return a + b;
  }

  public static void main(String args[])
  {
     try {
       Class cls = Class.forName("Main");
       Class partypes[] = new Class[2];
        partypes[0] = Integer.TYPE;
        partypes[1] = Integer.TYPE;
        Method meth = cls.getMethod(
          "add", partypes);
        Main methobj = new Main();
        Object arglist[] = new Object[2];
        arglist[0] = new Integer(37);
        arglist[1] = new Integer(47);
        Object retobj
          = meth.invoke(methobj, arglist);
        Integer retval = (Integer)retobj;
        System.out.println(retval.intValue());
     }
     catch (Throwable e) {
        System.err.println(e);
     }
  }
}
\end{lstlisting}
apelează metoda add în mod dinamic.

Pentru a putea implementa aplicația, am presupus că în programele pe care le
optimizăm reflecția nu este utilizată pentru a apela metode dinamic.

\subsection{Invocarea dinamică de metode}

În Java 7, a fost introdusă o noua instrucțiune de cod mașină menită să
faciliteze implementarea de limbaje dinamice în Java \cite{java_invokedynamic}.
Aceasta este o alternativă mai rapidă la reflecție pentru apelarea de metode
care nu sunt cunoscute decât la rulare.

Din aceleași motive ca la reflecție, voi presupune că programele nu vor conține
această instrucțiune.

\subsubsection{Funcții lambda}

Funcțiile lambda (sau funcțiile anonime) sunt o adiție recentă în limbajul Java.
Implementarea folosește invocarea dinamică atunci când o astfel de funcție
este instanțiată.
Așadar, dacă un program folosește funcții anonime, acesta va încălca
presupunerile făcute și nu va putea fi optimizat corect.


\section{Structura lucrării}

În prima parte a acestei lucrări voi formaliza în mod teoretic
problema de a optimiza un program.
În a doua parte voi descrie limbajul Java, și modul de
funcționare al acestuia.
În a treia parte voi detalia cum putem adapta teoria de
optimizare pentru programe dezvoltate în limbajul Java.
În ultima parte voi prezenta detaliile de implementare ale
optimizatorului de Java.
