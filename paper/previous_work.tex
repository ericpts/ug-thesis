\section{Muncă anterioară}

\subsection{C si C++}

Compilatorul gcc este capabil să elimine funcțiile nefolosite,
însă acest comportament nu este implicit.

Pentru exemplificare, vom considera programul
\begin{lstlisting}[language=C, title=main.c, label=c_program]
#include <stdio.h>

void foo()
{
    puts("in foo!");
}

void bar()
{
    puts("in bar!");
}

int main()
{
    foo();
    return 0;
}
\end{lstlisting}

Dacă îl compilăm cu \textbf{gcc} folosind opțiunea \texttt{-Os}
(permite optimizările pentru dimensiune), atunci binarul generat
va conține atât funcția \texttt{foo}, care este apelată din
\texttt{main}, cât și funcția \texttt{bar}, care nu este
niciodată folosită.

\begin{lstlisting}[language=Bash]
$ gcc main.c -Os
$ nm a.out | grep "foo"
000000000000064a T foo
$ nm a.out | grep "bar"
0000000000000656 T bar
$
\end{lstlisting}

Acest lucru se datorează faptului că optimizatorul pentru C
lucrează cu simboluri, nu direct cu funcții.  Diferența este că
un simbol poate reprezenta atât o funcție, cât și o variabilă.
Așadar, când optimizatorul decide dacă să elimine un simbol,
acesta nu consideră dacă acel simbol este o funcție sau un obiect
(o variabilă se mai numește și obiect).

În limbajul C, un obiect global trebuie să fie inițializat cu o
valoare constantă (i.e., poate fi evaluată la compilare), ceea ce
înseamnă că inițializările nu pot avea efecte laterale
\cite{c_static_init}.

Pe de altă parte, în limbajul C++, inițializările pot conține
expresii arbitrare, care necesită evaluarea acestora la timpul
rulării \cite{cpp_static_init} și care pot avea efecte laterale.

Optimizarea de a elimina funcțiile nefolosite nu este activată în
mod implicit deoarece optimizatorul folosit în gcc lucrează și
pentru C, dar și pentru C++.

Această optimizare ar altera comportamentul unui program C++ care
folosește efecte laterale la inițializare, întrucât dacă
optimizatorul elimină din program un simbol care corespunde unei
variabile, acea variabilă nu mai este inițializată, la rularea
programului, iar efectul lateral corespunzător inițializării ei
nu se mai realizează.

\subsubsection{Opțiuni speciale date compilatorului}

Optimizarea de a elimina simboluri nefolosite nu este aplicabilă
oricărui program. Totodată, dacă programatorul îi garantează
compilatorului că programul nu folosește inițializări cu efecte
laterale, compilatorul gcc este capabil de a elimina funcții
nefolosite.

Vom folosi în continuare programul definit la \ref{c_program}.
Conform \cite{c_enable_optimization}, este necesar să pasăm
opțiuni suplimentare compilatorului, iar acesta va elimina
funcția nefolosită:

\begin{lstlisting}[language=Bash]
$ gcc main.c -Wl,-static -Wl,--gc-sections -fdata-sections -ffunction-sections -Os
$ nm a.out | grep "foo"
0000000000400acd T foo
$ nm a.out | grep "bar"
$
\end{lstlisting}

În concluzie, pentru limbajele C și C++, este nevoie ca
programatorul să garanteze că aplicarea optimizării păstrează
corectitudinea programului.

\subsection{Java}

\subsubsection{La compilare}

Compilatorul standard de Java, \texttt{javac}, nu este un compilator
optimizator: acesta doar transformă codul Java în bytecode, fără a aplica
transformări de optimizre pe acesta.

În limbajul Java compilatorul lucrează cu câte un fișier o data: acesta nu are
conceptul de proiect sau executabil, ci doar de fișiere clasă.  Compilatorul nu
cunoaște ce funcții pot fi apelate, sau din ce locuri, deci nu poate efectua
nicio optimizare.

\subsubsection{La Rulare}

Optimizarea Java se produce la timpul de rulare.  Mașina virtuală Java
interpretează codul binar, iar secțiunile care sunt executate foarte des (de
exemplu, corpul unei bucle, sau o metodă care este apelată la intervale
regulate) sunt optimizate direct la rulare.

Fiecare implementare a mașinii virtuale are propriile moduri de a optimiza
codul, însă cele mai moderne se bazează pe profilarea acestuia și detectarea
posibililor candidați pentru optimizare \cite{arnold2006dynamic}.

\subsubsection{Android}

Deși codul Java arbitrar este imposibil de optimizat, structuri
particulare de proiecte permit eliminarea de cod nefolosit. De-a lungul
timpului, au fost create mai multe optimizatoare pentru scopuri specifice, însă
majoritatea nu mai sunt întreținute în mod activ \cite{java_android_optimizers}.
Această lipsă de interes probabil se datorează faptului că cei mai mulți oameni
sunt deserviți foarte bine de optimizarea oferită de mașinile virtuale la timpul
de rulare.

În zilele noastre, cel mai întâlnit loc unde este JVM-ul nu este satisfăcător
este optimizarea de aplicații pentru sistemul de operare Android
\cite{proguard_google}.
Programul recomandat de Google este \texttt{Proguard} \cite{proguard}.
Deși Proguard este el capabil de optimizări de dimensiune, principalul
său scop este obfuscarea codului, pentru a nu putea fi descifrat (Eng. "reverse
engineered")

Soluția acestui program pentru a rezolva problema pe care o
întâmpină \texttt{javac} este să îi ceară programatorului să
specifice în mod explicit modurile în care programul este rulat
(e.g., de unde poate începe execuția).

Având această informație, \texttt{proguard} poate analiza static
proiectul pentru a deduce care metode pot fi eliminate.
