%% (Master) Thesis template
% Template version used: v1.4
%
% Largely adapted from Adrian Nievergelt's template for the ADPS
% (lecture notes) project.


%% We use the memoir class because it offers a many easy to use features.
\documentclass[11pt,a4paper,titlepage]{memoir}

%% Packages
%% ========

%% LaTeX Font encoding -- DO NOT CHANGE
\usepackage[OT1]{fontenc}

%% Babel provides support for languages.  'english' uses British
%% English hyphenation and text snippets like "Figure" and
%% "Theorem". Use the option 'ngerman' if your document is in German.
%% Use 'american' for American English.  Note that if you change this,
%% the next LaTeX run may show spurious errors.  Simply run it again.
%% If they persist, remove the .aux file and try again.
\usepackage[english]{babel}

%% Input encoding 'utf8'. In some cases you might need 'utf8x' for
%% extra symbols. Not all editors, especially on Windows, are UTF-8
%% capable, so you may want to use 'latin1' instead.
\usepackage[utf8]{inputenc}

%% This changes default fonts for both text and math mode to use Herman Zapfs
%% excellent Palatino font.  Do not change this.
\usepackage[sc]{mathpazo}

%% The AMS-LaTeX extensions for mathematical typesetting.  Do not
%% remove.
\usepackage{amsmath,amssymb,amsfonts,mathrsfs}

%% NTheorem is a reimplementation of the AMS Theorem package. This
%% will allow us to typeset theorems like examples, proofs and
%% similar.  Do not remove.
%% NOTE: Must be loaded AFTER amsmath, or the \qed placement will
%% break
\usepackage[amsmath,thmmarks]{ntheorem}

%% LaTeX' own graphics handling
\usepackage{graphicx}

%% We unfortunately need this for the Rules chapter.  Remove it
%% afterwards; or at least NEVER use its underlining features.
\usepackage{soul}

%% This allows you to add .pdf files. It is used to add the
%% declaration of originality.
\usepackage{pdfpages}

%% Some more packages that you may want to use.  Have a look at the
%% file, and consult the package docs for each.
%% See the TeXed file for more explanations

%% [OPT] Multi-rowed cells in tabulars
%\usepackage{multirow}

%% [REC] Intelligent cross reference package. This allows for nice
%% combined references that include the reference and a hint to where
%% to look for it.
\usepackage{varioref}

%% [OPT] Easily changeable quotes with \enquote{Text}
%\usepackage[german=swiss]{csquotes}

%% [REC] Format dates and time depending on locale
\usepackage{datetime}

%% [OPT] Provides a \cancel{} command to stroke through mathematics.
%\usepackage{cancel}

%% [NEED] This allows for additional typesetting tools in mathmode.
%% See its excellent documentation.
\usepackage{mathtools}

%% [ADV] Conditional commands
%\usepackage{ifthen}

%% [OPT] Manual large braces or other delimiters.
%\usepackage{bigdelim, bigstrut}

%% [REC] Alternate vector arrows. Use the command \vv{} to get scaled
%% vector arrows.
\usepackage[h]{esvect}

%% [NEED] Some extensions to tabulars and array environments.
\usepackage{array}

%% [OPT] Postscript support via pstricks graphics package. Very
%% diverse applications.
%\usepackage{pstricks,pst-all}

%% [?] This seems to allow us to define some additional counters.
%\usepackage{etex}

%% [ADV] XY-Pic to typeset some matrix-style graphics
%\usepackage[all]{xy}

%% [OPT] This is needed to generate an index at the end of the
%% document.
%\usepackage{makeidx}

%% [OPT] Fancy package for source code listings.  The template text
%% needs it for some LaTeX snippets; remove/adapt the \lstset when you
%% remove the template content.
\usepackage{listings}
\lstset{language=TeX,basicstyle={\normalfont\ttfamily}}

%% [REC] Fancy character protrusion.  Must be loaded after all fonts.
\usepackage[activate]{pdfcprot}

%% [REC] Nicer tables.  Read the excellent documentation.
\usepackage{booktabs}


%% Some extra formatting, esp. for code.
\usepackage{color}
\usepackage{fancyvrb}
\newcommand{\VerbBar}{|}
\newcommand{\VERB}{\Verb[commandchars=\\\{\}]}
\DefineVerbatimEnvironment{Highlighting}{Verbatim}{commandchars=\\\{\}}

\newenvironment{Shaded}{}{}
\newcommand{\KeywordTok}[1]{\textcolor[rgb]{0.00,0.44,0.13}{\textbf{#1}}}
\newcommand{\DataTypeTok}[1]{\textcolor[rgb]{0.56,0.13,0.00}{#1}}
\newcommand{\DecValTok}[1]{\textcolor[rgb]{0.25,0.63,0.44}{#1}}
\newcommand{\BaseNTok}[1]{\textcolor[rgb]{0.25,0.63,0.44}{#1}}
\newcommand{\FloatTok}[1]{\textcolor[rgb]{0.25,0.63,0.44}{#1}}
\newcommand{\ConstantTok}[1]{\textcolor[rgb]{0.53,0.00,0.00}{#1}}
\newcommand{\CharTok}[1]{\textcolor[rgb]{0.25,0.44,0.63}{#1}}
\newcommand{\SpecialCharTok}[1]{\textcolor[rgb]{0.25,0.44,0.63}{#1}}
\newcommand{\StringTok}[1]{\textcolor[rgb]{0.25,0.44,0.63}{#1}}
\newcommand{\VerbatimStringTok}[1]{\textcolor[rgb]{0.25,0.44,0.63}{#1}}
\newcommand{\SpecialStringTok}[1]{\textcolor[rgb]{0.73,0.40,0.53}{#1}}
\newcommand{\ImportTok}[1]{#1}
\newcommand{\CommentTok}[1]{\textcolor[rgb]{0.38,0.63,0.69}{\textit{#1}}}
\newcommand{\DocumentationTok}[1]{\textcolor[rgb]{0.73,0.13,0.13}{\textit{#1}}}
\newcommand{\AnnotationTok}[1]{\textcolor[rgb]{0.38,0.63,0.69}{\textbf{\textit{#1}}}}
\newcommand{\CommentVarTok}[1]{\textcolor[rgb]{0.38,0.63,0.69}{\textbf{\textit{#1}}}}
\newcommand{\OtherTok}[1]{\textcolor[rgb]{0.00,0.44,0.13}{#1}}
\newcommand{\FunctionTok}[1]{\textcolor[rgb]{0.02,0.16,0.49}{#1}}
\newcommand{\VariableTok}[1]{\textcolor[rgb]{0.10,0.09,0.49}{#1}}
\newcommand{\ControlFlowTok}[1]{\textcolor[rgb]{0.00,0.44,0.13}{\textbf{#1}}}
\newcommand{\OperatorTok}[1]{\textcolor[rgb]{0.40,0.40,0.40}{#1}}
\newcommand{\BuiltInTok}[1]{#1}
\newcommand{\ExtensionTok}[1]{#1}
\newcommand{\PreprocessorTok}[1]{\textcolor[rgb]{0.74,0.48,0.00}{#1}}
\newcommand{\AttributeTok}[1]{\textcolor[rgb]{0.49,0.56,0.16}{#1}}
\newcommand{\RegionMarkerTok}[1]{#1}
\newcommand{\InformationTok}[1]{\textcolor[rgb]{0.38,0.63,0.69}{\textbf{\textit{#1}}}}
\newcommand{\WarningTok}[1]{\textcolor[rgb]{0.38,0.63,0.69}{\textbf{\textit{#1}}}}
\newcommand{\AlertTok}[1]{\textcolor[rgb]{1.00,0.00,0.00}{\textbf{#1}}}
\newcommand{\ErrorTok}[1]{\textcolor[rgb]{1.00,0.00,0.00}{\textbf{#1}}}
\newcommand{\NormalTok}[1]{#1}


%% Our layout configuration.  DO NOT CHANGE.
%% Memoir layout setup

%% NOTE: You are strongly advised not to change any of them unless you
%% know what you are doing.  These settings strongly interact in the
%% final look of the document.

% Dependencies
\usepackage{ETHlogo}

% Turn extra space before chapter headings off.
\setlength{\beforechapskip}{0pt}

\nonzeroparskip
\parindent=0pt
\defaultlists

% Chapter style redefinition
\makeatletter

\if@twoside
  \pagestyle{Ruled}
  \copypagestyle{chapter}{Ruled}
\else
  \pagestyle{ruled}
  \copypagestyle{chapter}{ruled}
\fi
\makeoddhead{chapter}{}{}{}
\makeevenhead{chapter}{}{}{}
\makeheadrule{chapter}{\textwidth}{0pt}
\copypagestyle{abstract}{empty}

\makechapterstyle{bianchimod}{%
  \chapterstyle{default}
  \renewcommand*{\chapnamefont}{\normalfont\Large\sffamily}
  \renewcommand*{\chapnumfont}{\normalfont\Large\sffamily}
  \renewcommand*{\printchaptername}{%
    \chapnamefont\centering\@chapapp}
  \renewcommand*{\printchapternum}{\chapnumfont {\thechapter}}
  \renewcommand*{\chaptitlefont}{\normalfont\huge\sffamily}
  \renewcommand*{\printchaptertitle}[1]{%
    \hrule\vskip\onelineskip \centering \chaptitlefont\textbf{\vphantom{gyM}##1}\par}
  \renewcommand*{\afterchaptertitle}{\vskip\onelineskip \hrule\vskip
    \afterchapskip}
  \renewcommand*{\printchapternonum}{%
    \vphantom{\chapnumfont {9}}\afterchapternum}}

% Use the newly defined style
\chapterstyle{bianchimod}

\setsecheadstyle{\Large\bfseries\sffamily}
\setsubsecheadstyle{\large\bfseries\sffamily}
\setsubsubsecheadstyle{\bfseries\sffamily}
\setparaheadstyle{\normalsize\bfseries\sffamily}
\setsubparaheadstyle{\normalsize\itshape\sffamily}
\setsubparaindent{0pt}

% Set captions to a more separated style for clearness
\captionnamefont{\sffamily\bfseries\footnotesize}
\captiontitlefont{\sffamily\footnotesize}
\setlength{\intextsep}{16pt}
\setlength{\belowcaptionskip}{1pt}

% Set section and TOC numbering depth to subsection
\setsecnumdepth{subsection}
\settocdepth{subsection}

%% Titlepage adjustments
\pretitle{\vspace{0pt plus 0.7fill}\begin{center}\HUGE\sffamily\bfseries}
\posttitle{\end{center}\par}
\preauthor{\par\begin{center}\let\and\\\Large\sffamily}
\postauthor{\end{center}}
\predate{\par\begin{center}\Large\sffamily}
\postdate{\end{center}}

\def\@advisors{}
\newcommand{\advisors}[1]{\def\@advisors{#1}}
\def\@department{}
\newcommand{\department}[1]{\def\@department{#1}}
\def\@thesistype{}
\newcommand{\thesistype}[1]{\def\@thesistype{#1}}

\renewcommand{\maketitlehooka}{\noindent\ETHlogo[2in]}

\renewcommand{\maketitlehookb}{\vspace{1in}%
  \par\begin{center}\Large\sffamily\@thesistype\end{center}}

\renewcommand{\maketitlehookd}{%
  \vfill\par
  \begin{flushright}
    \sffamily
    \@advisors\par
    \@department, UNIBUC
  \end{flushright}
}

\checkandfixthelayout

\setlength{\droptitle}{-48pt}

\makeatother

% This defines how theorems should look. Best leave as is.
\theoremstyle{plain}
\setlength\theorempostskipamount{0pt}

%%% Local Variables:
%%% mode: latex
%%% TeX-master: "thesis"
%%% End:


%% Theorem environments.  You will have to adapt this for a German
%% thesis.
%% Theorem-like environments

%% This can be changed according to language. You can comment out the ones you
%% don't need.

\numberwithin{equation}{chapter}

%% German theorems
%\newtheorem{satz}{Satz}[chapter]
%\newtheorem{beispiel}[satz]{Beispiel}
%\newtheorem{bemerkung}[satz]{Bemerkung}
%\newtheorem{korrolar}[satz]{Korrolar}
%\newtheorem{definition}[satz]{Definition}
%\newtheorem{lemma}[satz]{Lemma}
%\newtheorem{proposition}[satz]{Proposition}

%% English variants
\newtheorem{theorem}{Teoremă}[chapter]
\newtheorem{example}[theorem]{Exemplu}
\newtheorem{remark}[theorem]{Remarcă}
\newtheorem{corollary}[theorem]{Corolar}
\newtheorem{definition}[theorem]{Definiție}
\newtheorem{lemma}[theorem]{Lemă}
\newtheorem{proposition}[theorem]{Propozție}

%% Proof environment with a small square as a "qed" symbol
\theoremstyle{nonumberplain}
\theorembodyfont{\normalfont}
\theoremsymbol{\ensuremath{\square}}
\newtheorem{proof}{Proof}
%\newtheorem{beweis}{Beweis}


%% Helpful macros.
%% Custom commands
%% ===============

%% Special characters for number sets, e.g. real or complex numbers.
\newcommand{\C}{\mathbb{C}}
\newcommand{\K}{\mathbb{K}}
\newcommand{\N}{\mathbb{N}}
\newcommand{\Q}{\mathbb{Q}}
\newcommand{\R}{\mathbb{R}}
\newcommand{\Z}{\mathbb{Z}}
\newcommand{\X}{\mathbb{X}}

%% Fixed/scaling delimiter examples (see mathtools documentation)
\DeclarePairedDelimiter\abs{\lvert}{\rvert}
\DeclarePairedDelimiter\norm{\lVert}{\rVert}

%% Use the alternative epsilon per default and define the old one as \oldepsilon
\let\oldepsilon\epsilon
\renewcommand{\epsilon}{\ensuremath\varepsilon}

%% Also set the alternate phi as default.
\let\oldphi\phi
\renewcommand{\phi}{\ensuremath{\varphi}}


%% Make document internal hyperlinks wherever possible. (TOC, references)
%% This MUST be loaded after varioref, which is loaded in 'extrapackages'
%% above.  We just load it last to be safe.
\usepackage[linkcolor=black,colorlinks=true,citecolor=black,filecolor=black]{hyperref}


%% Document information
%% ====================

\title{Un optimizator de dimensiune pentru Java}
\author{Petru-Eric Stavarache}
\thesistype{Teza de Licenta}
\advisors{Coordonator Prof.\ Dr. Traian-Florin Serbanuta}
\department{Facultatea de Matematica si Informatica}
\date{4 Iunie, 2018}

\begin{document}

\frontmatter

%% Title page is autogenerated from document information above.  DO
%% NOT CHANGE.
\begin{titlingpage}
  \calccentering{\unitlength}
  \begin{adjustwidth*}{\unitlength-24pt}{-\unitlength-24pt}
    \maketitle
  \end{adjustwidth*}
\end{titlingpage}

%% The abstract of your thesis.  Edit the file as needed.
\begin{abstract}
	Sistemul de operare Android este cea mai populara platforma pentru telefoanele
	mobile, iar limbajul utilizat pentru a dezvolta aplicatii este Java.
	O reducere de doar cativa octeti a dimensiunii pachetului unei aplicatii
	populare, precum Facebook, ar duce la economisirea de cativa giga-octeti de
	trafic de internet lunar.

	In aceasta teza am proiectat si implementat un compilator care efectueaza
	optimizari de dimensiune a codului asupra fisierelor compilate Java.

	Aceste optimizari elimina functiile si metodele neutilizate dintr-un proiect
	dezvoltat in limbajul Java si se bazeaza pe analiza statica a proiectului.
	O serie de teste au fost create pentru a testa corectitudinea, cat si eficienta
	optimizarilor aplicate.
	Compilatorul lucreaza direct cu fisiere compilate, in formatul utilizat si de
	catre Android.

\end{abstract}


%% TOC with the proper setup, do not change.
\cleartorecto
\tableofcontents
\mainmatter

%% Your real content!
\chapter{Introducere}

Eliminarea codului inutil (Eng. "Dead code elimination")
\cite{wiki:deadcodeelimination} este o optimizare clasică.  Ea
presupune eliminarea dintr-un program a codului care nu afectează
rezultatul computației.

Codul poate să fie eliminat dacă, de exemplu, nu există niciun fir
de execuție care să conțină instrucțiunile respective, sau dacă
acel cod nu are efecte laterale.

Acest tip de optimizare este implementată tradițional în
limbajele compilate, precum C, cât și în cele `compilate-la-timp'
(Eng. `just-in-time') precum Java sau JavaScript.  Beneficiile
principale aduse de către optimizare sunt:

\begin{enumerate}
	\item Reducerea dimensiunii programului
	\item Creșterea vitezei de execuție
\end{enumerate}

Cu cât un limbaj este mai dinamic, și permite mai multe schimbări ale
comportamentului la rulare, cu atât eliminarea de cod nefolosit
devine o sarcină mai grea.

De exemplu, echipa care dezvoltă motorul de JavaScript pentru
browser-ul Internet Explorer a avut probleme de corectitudine în
optimizările realizate, datorită naturii dinamice a limbajului
JavaScript \cite{deadcodeeliminationforbeginners}.

Deși pentru un programator nu este natural să creeze cod
nefolosit, acest lucru nu înseamnă ca principiul eliminării
acestuia nu poate fi aplicat; compilatoarele în sine, prin modul
lor de a trece de mai multe ori prin codul sursă și de a avea mai
multe reprezentări intermediare, pot genera cod nefolosit.

Limbajul C, prin includerea de fișiere mot-a-mot, este un exemplu
bun pentru acest lucru: un program de câteva linii, care afișează
"Hello, world!", ajunge să aibă, înaintea eliminării codului mort,
câteva mii de linii și sute de funcții nefolosite. Acest lucru se datorează
nevoii includerii bibliotecii standard.

\section{Eliminarea metodelor}

Eliminarea metodelor și funcțiilor este un una dintre multele
posibilități de a scăpa de cod nefolosit.

Procedeul constă în îndepărtarea dintr-un program a funcțiilor și a
metodelor care nu sunt niciodată apelate.

Această lucrare va explora această optimizare particulară, atât
teoretic, cât și implementat în limbajul Java.

\section{Muncă anterioară}

\subsection{C si C++}

Compilatorul gcc este capabil să elimine funcțiile nefolosite,
însă acest comportament nu este default.

Pentru exemplificare, vom considera programul
\begin{lstlisting}[language=C, title=main.c,
label=c:program]
#include <stdio.h>

void foo()
{
    puts("in foo!");
}

void bar()
{
    puts("in bar!");
}

int main()
{
    foo();
    return 0;
}
\end{lstlisting}

Dacă îl compilăm cu \textbf{gcc} folosind opțiunea \texttt{-Os}
(permite optimizările pentru dimensiune), atunci binarul generat
va conține atât funcția \texttt{foo}, care este apelată din
\texttt{main}, cât și funcția \texttt{bar}, care nu este
niciodată folosită.

\begin{lstlisting}[language=Bash]
$ gcc main.c -Os
$ nm a.out | grep "foo"
000000000000064a T foo
$ nm a.out | grep "bar"
0000000000000656 T bar
$
\end{lstlisting}

Acest lucru se datorează faptului că optimizatorul pentru C
lucrează cu simboluri, nu direct cu funcții.  Diferența este că
un simbol poate reprezenta atât o funcție, cât și o variabilă.
Așadar, când optimizatorul decidă dacă să elimine un simbol,
acesta nu consideră dacă acel simbol este o funcție sau un obiect
(o variabilă se mai numește și obiect).

În limbajul C, un obiect global trebuie să fie inițializat cu o
valoare constantă (i.e., poate fi evaluată la compilare), ceea ce
înseamnă că inițializările nu pot avea efecte laterale
\cite{c_static_init}.

Pe de altă parte, în limbajul C++, inițializările pot conține
expresii arbitrare, care necesită evaluarea acestora la timpul
rulării \cite{cpp_static_init} și care pot avea efecte laterale.

Optimizarea de a elimina funcțiile nefolosite nu este activată în
mod implicit deoarece optimizatorul folosit în gcc lucrează și
pentru C, dar și pentru C++.

Această optimizare ar altera comportamentul unui program C++ care
folosește efecte laterale la inițializare, întrucât dacă
optimizatorul elimină din program un simbol care corespunde unei
variabile, acea variabilă nu mai este inițializată, la rularea
programului, iar efectul lateral corespunzător inițializării ei
nu se mai realizează.

\subsubsection{Opțiuni speciale date compilatorului}

Optimizarea de a elimina simboluri nefolosite nu este aplicabilă
oricărui program. Totodată, dacă programatorul îi garantează
compilatorului că programul nu folosește inițializări cu efecte
laterale, compilatorul gcc este capabil de a elimina funcții
nefolosite.

Vom folosi în continuare programul definit la \ref{c:program}.
Conform \cite{c_enable_optimization}, este necesar să pasăm
opțiuni suplimentare compilatorului, iar acesta va elimina
funcția nefolosită:

\begin{lstlisting}[language=Bash]
$ gcc main.c -Wl,-static -Wl,--gc-sections -fdata-sections -ffunction-sections -Os
$ nm a.out | grep "foo"
0000000000400acd T foo
$ nm a.out | grep "bar"
$
\end{lstlisting}

În concluzie, pentru limbajele C și C++, este nevoie ca
programatorul să garanteze că aplicarea optimizării păstrează
corectitudinea programului.

\subsection{Java}

Compilatorul standard de Java, \texttt{javac}, este incapabil să
efectueaze optimizarea de a elimina metode nefolosite.

În limbajul C, compilatorul gcc știe că programul își începe
execuția din funcția \texttt{main}, și că acesta este singurul
mod în care programul poate fi folosit.

În limbajul Java, compilatorul \texttt{javac}, lucrează cu câte
un fișier o data: acesta nu are conceptul de proiect sau
executabil, ci doar de fișiere clasă.
Compilatorul nu cunoaște ce funcții pot fi apelate, sau
din ce locuri, deci nu poate efectua nicio optimizare.

\subsubsection{Android}

Deși codul Java arbitrar este imposibil de optimizat, structuri
particulare de proiecte permit eliminarea de cod nefolosit.

Acest lucru este folosit pentru aplicațiile dezvoltate pentru
sistemul de operare Android \cite{android_proguard}.
Programul ales pentru a realiza sarcina de optimizare este
\texttt{Proguard} \cite{proguard}.

Soluția acestui program pentru a rezolva problema pe care o
întămpină \texttt{javac} este să îi ceară programatorului să
specifice în mod explicit modurile în care programul este rulat
(e.g., de unde poate începe execuția).

Având această informație, \texttt{proguard} poate analiza static
proiectul pentru a deduce care metode pot fi eliminate.


\section{Asumpții}

Înainte de a începe descrierea problemei, voi expune asumpțiile pe care le-am
făcut în implementarea aplicației pentru Java.
Deși aceste asumpții par destul de restrictive, cele mai multe proiecte normale,
atât de Android, cât și normale, le vor satisface.

Programul dezvoltat în această lucrare optimizează programele Java la compilare.
În urmare, acesta nu poate să trateze invocarea de metode la rulare.

Programul va emite un mesaj de eroare în caz că detectează că asumpția este
încălcată,

\subsection{Reflecția}

Reflecția constă în introspecția programului la rulare -- spre exemplu,
identificarea câmpurilor sau metodelor unei clase.

Deși reflecția nu este de obicei folosită în acest scop (decât în circumstanțe
specifice, de exemplu implementarea altor limbaje), aceasta permite programului
să apeleze metode care nu sunt cunoscute decât la timpul rulării.

De exemplu, programul \ref{java_reflection}
\begin{lstlisting}[language=Java]
public class Main {
  public int add(int a, int b)
  {
     return a + b;
  }

  public static void main(String args[])
  {
     try {
       Class cls = Class.forName("Main");
       Class partypes[] = new Class[2];
        partypes[0] = Integer.TYPE;
        partypes[1] = Integer.TYPE;
        Method meth = cls.getMethod(
          "add", partypes);
        Main methobj = new Main();
        Object arglist[] = new Object[2];
        arglist[0] = new Integer(37);
        arglist[1] = new Integer(47);
        Object retobj
          = meth.invoke(methobj, arglist);
        Integer retval = (Integer)retobj;
        System.out.println(retval.intValue());
     }
     catch (Throwable e) {
        System.err.println(e);
     }
  }
}
\end{lstlisting}
apelează metoda add în mod dinamic.

Pentru a putea implementa aplicația, am presupus că în programele pe care le
optimizăm reflecția nu este utilizată pentru a apela metode dinamic.

\subsection{Invocarea dinamică de metode}

În Java 7, a fost introdusă o noua instrucțiune de cod mașină menită să
faciliteze implementarea de limbaje dinamice în Java \cite{java_invokedynamic}.
Aceasta este o alternativă mai rapidă la reflecție pentru apelarea de metode
care nu sunt cunoscute decât la rulare.

Din aceleași motive ca la reflecție, voi presupune că programele nu vor conține
această instrucțiune.

\subsubsection{Funcții lambda}

Funcțiile lambda (sau funcțiile anonime) sunt o adiție recentă în limbajul Java.
Implementarea folosește invocarea dinamică atunci când o astfel de funcție
este instanțiată.
Așadar, dacă un program folosește funcții anonime, acesta va încălca
presupunerile făcute și nu va putea fi optimizat corect.


\section{Structura lucrării}

În prima parte a acestei lucrări voi formaliza în mod teoretic
problema de a optimiza un program.
În a doua parte voi descrie limbajul Java, și modul de
funcționare al acestuia.
În a treia parte voi detalia cum putem adapta teoria de
optimizare pentru programe dezvoltate în limbajul Java.
În ultima parte voi prezenta detaliile de implementare ale
optimizatorului de Java.

%% \chapter{Writing scientific texts in English}

This chapter was originally a separate document written by Reto
Spöhel.  It is reprinted here so that the template can serve as a
quick guide to thesis writing, and to provide some more example
material to give you a feeling for good typesetting.

% We're going to need an extra theorem-like environment for this
% chapter
\theoremstyle{plain}
\theoremsymbol{}
\newtheorem{Rule}[theorem]{Rule}

\section{Basic writing rules}

The following rules need little further explanation; they are best
understood by looking at the example in the booklet by Knuth et al.,
§2--§3.

\begin{Rule}
	Write texts, not chains of formulas.
\end{Rule}

More specifically, write full sentences that are logically
interconnected by phrases like `Therefore', `However', `On the other
hand', etc.\ where appropriate.

\begin{Rule}
	Displayed formulas should be embedded in your text and punctuated
	with it.
\end{Rule}

In other words, your writing should not be divided into `text parts'
and `formula parts'; instead the formulas should be tied together by
your prose such that there is a natural flow to your writing.

\section{Being nice to the reader}

Try to write your text in such a way that a reader enjoys reading
it. That's of course a lofty goal, but nevertheless one you should
aspire to!

\begin{Rule}
	Be nice to the reader.
\end{Rule}

Give some intuition or easy example for definitions and theorems which
might be hard to digest. Remind the reader of notations you introduced
many pages ago -- chances are he has forgotten them. Illustrate your
writing with diagrams and pictures where this helps the reader. Etc.

\begin{Rule}
	Organize your writing.
\end{Rule}

Think carefully about how you subdivide your thesis into chapters,
sections, and possibly subsections.  Give overviews at the beginning
of your thesis and of each chapter, so the reader knows what to
expect. In proofs, outline the main ideas before going into technical
details. Give the reader the opportunity to `catch up with you' by
summing up your findings periodically.

\emph{Useful phrases:} `So far we have shown that \ldots', `It remains
to show that \ldots', `Recall that we want to prove inequality (7), as
this will allow us to deduce that \ldots', `Thus we can conclude that
\ldots. Next, we would like to find out whether \ldots', etc.

\begin{Rule}
	Don't say the same thing twice without telling the reader that you
	are saying it twice.
\end{Rule}

Repetition of key ideas is important and helpful. However, if you
present the same idea, definition or observation twice (in the same or
different words) without telling the reader, he will be looking for
something new where there is nothing new.

\emph{Useful phrases:} `Recall that [we have seen in Chapter 5 that]
\ldots', `As argued before / in the proof of Lemma 3, \ldots', `As
mentioned in the introduction, \ldots', `In other words, \ldots', etc.

\begin{Rule}
	Don't make statements that you will justify later without telling
	the reader that you will justify them later.
\end{Rule}

This rule also applies when the justification is coming right in the
next sentence!  The reasoning should be clear: if you violate it, the
reader will lose valuable time trying to figure out on his own what
you were going to explain to him anyway.

\emph{Useful phrases:} `Next we argue that \ldots', `As we shall see,
\ldots', `We will see in the next section that \ldots, etc.

\section{A few important grammar rules}

\begin{Rule}
	\label{rule:no-comma-before-that}
	There is (almost) \emph{never} a comma before `that'.
\end{Rule}

It's really that simple. Examples:
\begin{quote}
	We assume that \ldots\\
	\emph{Wir nehmen an, dass \ldots}

	It follows that \ldots\\
	\emph{Daraus folgt, dass \ldots}

	`thrice' is a word that is seldom used.\\
	\emph{`thrice' ist ein Wort, das selten verwendet wird.}
\end{quote}
Exceptions to this rule are rare and usually pretty obvious. For
example, you may end up with a comma before `that' because `i.e.' is
spelled out as `that is':
\begin{quote}
	For \(p(n)=\log n/n\) we have \ldots{} However, if we choose \(p\) a
	little bit higher, that is \(p(n)=(1+\varepsilon)\log n/n\) for some
	\(\varepsilon>0\), we obtain that\ldots
\end{quote}
Or you may get a comma before `that' because there is some additional
information inserted in the middle of your sentence:
\begin{quote}
	Thus we found a number, namely \(n_0\), that satisfies equation (13).
\end{quote}
If the additional information is left out, the sentence has no comma:
\begin{quote}
	Thus we found a number that satisfies equation (13).
\end{quote}
(For `that' as a relative pronoun, see also
Rules~\ref{rule:non-defining-has-comma}
and~\ref{rule:defining-without-comma} below.)

\begin{Rule}
	There is usually no comma before `if'.
\end{Rule}

Example:
\begin{quote}
	A graph is not \(3\)-colorable if it contains a \(4\)-clique.\\
	\emph{Ein Graph ist nicht \(3\)-färbbar, wenn er eine \(4\)-Clique
		enthält.}
\end{quote}
However, if the `if' clause comes first, it is usually separated from
the main clause by a comma:
\begin{quote}
	If a graph contains a \(4\)-clique, it is not \(3\)-colorable .\\
	\emph{Wenn ein Graph eine \(4\)-Clique enthält, ist er nicht
		\(3\)-färbbar.}
\end{quote}

There are more exceptions to these rules than to
Rule~\ref{rule:no-comma-before-that}, which is why we are not
discussing them here. Just keep in mind: don't put a comma before `if'
without good reason.

\begin{Rule}
	\label{rule:non-defining-has-comma}
	Non-defining relative clauses have commas.
\end{Rule}
\begin{Rule}
	\label{rule:defining-without-comma}
	Defining relative clauses have no commas.
\end{Rule}

In English, it is very important to distinguish between two types of
relative clauses: defining and non-defining ones. This is a
distinction you absolutely need to understand to write scientific
texts, because mistakes in this area actually distort the meaning of
your text!

It's probably easier to explain first what a \emph{non-defining}
relative clause is. A non-defining relative clauses simply gives
additional information \emph{that could also be left out} (or given in
a separate sentence). For example, the sentence
\begin{quote}
	The \textsc{WeirdSort} algorithm, which was found by the famous
	mathematician John Doe, is theoretically best possible but difficult
	to implement in practice.
\end{quote}
would be fully understandable if the relative clause were left out
completely. It could also be rephrased as two separate sentences:
\begin{quote}
	The \textsc{WeirdSort} algorithm is theoretically best possible but
	difficult to implement in practice. [By the way,] \textsc{WeirdSort}
	was found by the famous mathematician John Doe.
\end{quote}
This is what a non-defining relative clause is. \emph{Non-defining
	relative clauses are always written with commas.} As a corollary we
obtain that you cannot use `that' in non-defining relative clauses
(see Rule~\ref{rule:no-comma-before-that}!). It would be wrong to
write
\begin{quote}
	\st{The \textsc{WeirdSort} algorithm, that was found by the famous
		mathematician John Doe, is theoretically best possible but
		difficult to implement in practice.}
\end{quote}
A special case that warrants its own example is when `which' is
referring to the entire preceding sentence:
\begin{quote}
	Thus inequality (7) is true, which implies that the Riemann
	hypothesis holds.
\end{quote}
As before, this is a non-defining relative sentence (it could be left
out) and therefore needs a comma.

So let's discuss \emph{defining} relative clauses next. A defining
relative clause tells the reader \emph{which specific item the main
	clause is talking about}. Leaving it out either changes the meaning
of the sentence or renders it incomprehensible altogether.  Consider
the following example:

\begin{quote}
	The \textsc{WeirdSort} algorithm is difficult to implement in
	practice. In contrast, the algorithm that we suggest is very simple.
\end{quote}

Here the relative clause `that we suggest' cannot be left out -- the
remaining sentence would make no sense since the reader would not know
which algorithm it is talking about. This is what a defining relative
clause is. \textit{Defining relative clauses are never written with
	commas.} Usually, you can use both `that' and `which' in defining
relative clauses, although in many cases `that' sounds better.

As a final example, consider the following sentence:
\begin{quote}
	For the elements in \(\mathcal{B}\) which satisfy property (A), we
	know that equation (37) holds.
\end{quote}
This sentence does not make a statement about all elements in
\(\mathcal{B}\), only about those satisfying property (A). The relative
clause is \emph{defining}. (Thus we could also use `that' in place of
`which'.)

In contrast, if we add a comma the sentence reads
\begin{quote}
	For the elements in \(\mathcal{B}\), which satisfy property (A), we
	know that equation (37) holds.
\end{quote}

Now the relative clause is \emph{non-defining} -- it just mentions in
passing that all elements in \(\mathcal{B}\) satisfy property (A). The
main clause states that equation (37) holds for \emph{all} elements in
\(\mathcal{B}\). See the difference?

\section[Things you (usually) don't say in English]%
 {Things you (usually) don't say in English -- and what to say
  instead}
\label{sec:list}

Table~\ref{tab:things-you-dont-say} lists some common mistakes and
alternatives.  The entries should not be taken as gospel -- they don't
necessarily mean that a given word or formulation is wrong under all
circumstances (obviously, this depends a lot on the context). However,
in nine out of ten instances the suggested alternative is the better
word to use.

\begin{table}
	\centering
	\caption{Things you (usually) don't say}
	\label{tab:things-you-dont-say}
	\begin{tabular}{lll}
		\toprule
		\st{It holds (that) \dots}                & We have \dots                             & \emph{Es gilt \dots}                              \\
		\multicolumn{3}{l}{\quad\footnotesize(`Equation (5) holds.' is fine, though.)}                                                            \\
		\st{$x$ fulfills property $\mathcal{P}$.} & \(x\) satisfies property \(\mathcal{P}\). & \emph{\(x\) erfüllt Eigenschaft \(\mathcal{P}\).} \\
		\st{in average}                           & on average                                & \emph{im Durchschnitt}                            \\
		\st{estimation}                           & estimate                                  & \emph{Abschätzung}                                \\
		\st{composed number}                      & composite number                          & \emph{zusammengesetzte Zahl}                      \\
		\st{with the help of}                     & using                                     & \emph{mit Hilfe von}                              \\
		\st{surely}                               & clearly                                   & \emph{sicher, bestimmt}                           \\
		\st{monotonously increasing}              & monotonically incr.                       & \emph{monoton steigend}                           \\
		\multicolumn{3}{l}{\quad\footnotesize(Actually, in most cases `increasing' is just fine.)}                                                \\
		\bottomrule
	\end{tabular}
\end{table}

%%% Local Variables:
%%% mode: latex
%%% TeX-master: "thesis"
%%% End:

%% \chapter{Typography}


\section{Punctuation}

\begin{Rule}
  Use opening (`) and closing (') quotation marks correctly.
\end{Rule}

In \LaTeX, the closing quotation mark is typed like a normal
apostrophe, while the opening quotation mark is typed using the French
\emph{accent grave} on your keyboard (the \emph{accent grave} is the
one going down, as in \emph{frère}).

Note that any punctuation that \emph{semantically} follows quoted
speech goes inside the quotes in American English, but outside in
Britain.  Also, Americans use double quotes first.  Oppose
\begin{quote}
  ``Using `lasers,' we punch a hole in \ldots\ the Ozone Layer,''
  Dr.\ Evil said.
\end{quote}
to
\begin{quote}
  `Using ``lasers'', we punch a hole in \ldots\ the Ozone Layer',
  Dr.\ Evil said.
\end{quote}

\begin{Rule}
  Use hyphens (-), en-dashes (--) and em-dashes (---) correctly.
\end{Rule}

A hyphen is only used in words like `well-known', `$3$-colorable'
etc., or to separate words that continue in the next line (which is
known as hyphenation).  It is entered as a single ASCII hyphen
character (\texttt{-}).

To denote ranges of numbers, chapters, etc., use an en-dash (entered
as two ASCII hyphens \texttt{--}) with no spaces on either side.  For
example, using Equations (1)--(3), we see\ldots

As the equivalent of the German \emph{Gedankenstrich}, use an en-dash
with spaces on both sides -- in the title of Section \ref{sec:list},
it would be wrong to use a hyphen instead of the dash. (Some English
authors use the even longer emdash (---) instead, which is typed as
three subsequent hyphens in \LaTeX. This emdash is used without spaces
around it---like so.)


\section{Spacing}

\begin{Rule}
  \label{rule:no-manual-spacing}
  Do not add spacing manually.
\end{Rule}

You should never use the commands \lstinline-\\- (except within
tabulars and arrays), \lstinline[showspaces=true]-\ - (except to
prevent a sentence-ending space after Dr.\ and such),
\lstinline-\vspace-, \lstinline-\hspace-, etc.  The choices programmed
into \LaTeX{} and this style should cover almost all cases.  Doing it
manually quickly leads to inconsistent spacing, which looks terrible.
Note that this list of commands is by no means conclusive.

\begin{Rule}
  Judiciously insert spacing in maths where it helps.
\end{Rule}

This directly contradicts Rule~\ref{rule:no-manual-spacing}, but in
some cases \TeX{} fails to correctly decide how much spacing is
required.  For example, consider
\begin{displaymath}
  f(a,b) = f(a+b, a-b).
\end{displaymath}
In such cases, inserting a thin math space \lstinline-\,- greatly
increases readability:
\begin{displaymath}
  f(a,b) = f(a+b,\, a-b).
\end{displaymath}

Along similar lines, there are variations of some symbols with
different spacing.  For example, Lagrange's Theorem states that
\(\abs{G}=[G:H]\abs{H}\), but the proof uses a bijection \(f\colon
aH\to bH\).  (Note how the first colon is symmetrically spaced, but
the second is not.)

\begin{Rule}
  Learn when to use \lstinline[showspaces=true]-\ - and
  \lstinline-\@-.
\end{Rule}

Unless you use `french spacing', the space at the end of a sentence is
slightly larger than the normal interword space.

The rule used by \TeX{} is that any space following a period,
exclamation mark or question mark is sentence-ending, except for
periods preceded by an upper-case letter.  Inserting \lstinline-\-
before a space turns it into an interword space, and inserting
\lstinline-\@- before a period makes it sentence-ending.  This means
you should write
\begin{lstlisting}
Prof.\ Dr.\ A. Steger is a member of CADMO\@.
If you want to write a thesis with her, you
should use this template.
\end{lstlisting}
which turns into
\begin{quote}
  Prof.\ Dr.\ A. Steger is a member of CADMO\@.  If you want to write
  a thesis with her, you should use this template.
\end{quote}
The effect becomes more dramatic in lines that are stretched slightly
during justification:
\begin{quote}
  \parbox{\linewidth}{\hbox to \linewidth{%
      Prof.\ Dr.\ A. Steger is a member of CADMO\@.  If you}}
\end{quote}

\begin{Rule}
  Place a non-breaking space (\lstinline-~-) right before references.
\end{Rule}

This is actually a slight simplification of the real rule, which
should invoke common sense.  Place non-breaking spaces where a line
break would look `funny' because it occurs right in the middle of a
construction, especially between a reference type (Chapter) and its
number.


\section{Choice of `fonts'}

Professional typography distinguishes many font attributes, such as
family, size, shape, and weight.  The choice for sectional divisions
and layout elements has been made, but you will still occasionally
want to switch to something else to get the reader's attention.  The
most important rule is very simple.

\begin{Rule}
  When emphasising a short bit of text, use \lstinline-\emph-.
\end{Rule}

In particular, \emph{never} use bold text (\lstinline-\textbf-).
Italics (or Roman type if used within italics) avoids distracting the
eye with the huge blobs of ink in the middle of the text that bold
text so quickly introduces.

Occasionally you will need more notation, for example, a consistent
typeface used to identify algorithms.

\begin{Rule}
  Vary one attribute at a time.
\end{Rule}

For example, for \textsc{WeirdSort} we only changed the shape to small
caps.  Changing two attributes, say, to bold small caps would be
excessive (\LaTeX{} does not even have this particular variation).
The same holds for mathematical notation: the reader can easily
distinguish \(g_n\), \(G(x)\), \(\mathcal{G}\) and \(\mathsf{G}\).

\begin{Rule}
  Never underline or uppercase.
\end{Rule}

No exceptions to this one, unless you are writing your thesis on a
typewriter.  Manually.  Uphill both ways.  In a blizzard.


\section{Displayed equations}

\begin{Rule}
  Insert paragraph breaks \emph{after} displays only where they
  belong.  Never insert paragraph breaks \emph{before} displays.
\end{Rule}

\LaTeX{} translates sequences of more than one linebreak (i.e., what
looks like an empty line in the source code) into a paragraph break in
almost all contexts.  This also happens before and after displays,
where extra spacing is inserted to give a visual indication of the
structure.  Adding a blank line in these places may look nice in the
sources, but compare the resulting display

\begin{displaymath}
  a = b
\end{displaymath}

to the following:
\begin{displaymath}
  a = b
\end{displaymath}
The first display is surrounded by blank lines, but the second is not.
It is bad style to start a paragraph with a display (you should always
tell the reader what the display means first), so the rule follows.

\begin{Rule}
  Never use \lstinline-eqnarray-.
\end{Rule}

It is at the root of most ill-spaced multiline displays.  The
\package{amsmath} package provides better alternatives, such as the
\lstinline-align- family
\begin{align*}
  f(x) &= \sin x, \\
  g(x) &= \cos x,
\end{align*}
and \lstinline-multline- which copes with excessively long equations:
\begin{multline*}
  \def\P{\mathrm P}
  \P\bigl[X_{t_0} \in (z_0, z_0+dz_0],\ldots, X_{t_n}\in(z_n,z_n+dz_n]\bigr]
  \\= \nu(dz_0) K_{t_1}(z_0,dz_1) K_{t_2-t_1}(z_1,dz_2)\cdots
  K_{t_n-t_{n-1}}(z_{n-1},dz_n).
\end{multline*}


\section{Floats}

By default this style provides floating environments for tables and
figures.  The general structure should be as follows:
\begin{lstlisting}
\begin{figure}
  \centering
  % content goes here
  \caption{A short caption}
  \label{some-short-label}
\end{figure}
\end{lstlisting}
Note that the label must follow the caption, otherwise the label will
refer to the surrounding section instead.  Also note that figures
should be captioned at the bottom, and tables at the top.

The whole point of floats is that they, well, \emph{float} to a place
where they fit without interrupting the text body.  This is a frequent
source of confusion and changes; please leave it as is.

\begin{Rule}
  Do not restrict float movement to only `here'
  \textnormal{(\lstinline-h-)}.
\end{Rule}

If you are still tempted, you should avoid the float altogether and
just show the figure or table inline, similar to a displayed equation.

%%% Local Variables:
%%% mode: latex
%%% TeX-master: "thesis"
%%% End:

%% \chapter{Example Chapter}

Dummy text.

\section{Example Section}

Dummy text.

\subsection{Example Subsection}

Dummy text.

\subsubsection{Example Subsubsection}

Dummy text.

\paragraph{Example Paragraph}

Dummy text.

\subparagraph{Example Subparagraph}

Dummy text.

%% This is an example first chapter.  You should put chapter/appendix that you
%% write into a separate file, and add a line \include{yourfilename} to
%% main.tex, where `yourfilename.tex' is the name of the chapter/appendix file.
%% You can process specific files by typing their names in at the
%% \files=
%% prompt when you run the file main.tex through LaTeX.
\chapter{Introduction}

Micro-optimization is a technique to reduce the overall operation count of
floating point operations.  In a standard floating point unit, floating
point operations are fairly high level, such as ``multiply'' and ``add'';
in a micro floating point unit ($\mu$FPU), these have been broken down into
their constituent low-level floating point operations on the mantissas and
exponents of the floating point numbers.

Chapter two describes the architecture of the $\mu$FPU unit, and the
motivations for the design decisions made.

Chapter three describes the design of the compiler, as well as how the
optimizations discussed in section~\ref{ch1:opts} were implemented.

Chapter four describes the purpose of test code that was compiled, and which
statistics were gathered by running it through the simulator.  The purpose
is to measure what effect the micro-optimizations had, compared to
unoptimized code.  Possible future expansions to the project are also
discussed.

\section{Motivations for micro-optimization}

The idea of micro-optimization is motivated by the recent trends in computer
architecture towards low-level parallelism and small, pipelineable
instruction sets \cite{patterson:risc,rad83}.  By getting rid of more
complex instructions and concentrating on optimizing frequently used
instructions, substantial increases in performance were realized.

Another important motivation was the trend towards placing more of the
burden of performance on the compiler.  Many of the new architectures depend
on an intelligent, optimizing compiler in order to realize anywhere near
their peak performance
\cite{ellis:bulldog,pet87,coutant:precision-compilers}.  In these cases, the
compiler not only is responsible for faithfully generating native code to
match the source language, but also must be aware of instruction latencies,
delayed branches, pipeline stages, and a multitude of other factors in order
to generate fast code \cite{gib86}.

Taking these ideas one step further, it seems that the floating point
operations that are normally single, large instructions can be further broken
down into smaller, simpler, faster instructions, with more control in the
compiler and less in the hardware.  This is the idea behind a
micro-optimizing FPU; break the floating point instructions down into their
basic components and use a small, fast implementation, with a large part of
the burden of hardware allocation and optimization shifted towards
compile-time.

Along with the hardware speedups possible by using a $\mu$FPU, there are
also optimizations that the compiler can perform on the code that is
generated.  In a normal sequence of floating point operations, there are
many hidden redundancies that can be eliminated by allowing the compiler to
control the floating point operations down to their lowest level.  These
optimizations are described in detail in section~\ref{ch1:opts}.

\section{Description of micro-optimization}\label{ch1:opts}

In order to perform a sequence of floating point operations, a normal FPU
performs many redundant internal shifts and normalizations in the process of
performing a sequence of operations.  However, if a compiler can
decompose the floating point operations it needs down to the lowest level,
it then can optimize away many of these redundant operations.

If there is some additional hardware support specifically for
micro-optimization, there are additional optimizations that can be
performed.  This hardware support entails extra ``guard bits'' on the
standard floating point formats, to allow several unnormalized operations to
be performed in a row without the loss information\footnote{A description of
the floating point format used is shown in figures~\ref{exponent-format}
and~\ref{mantissa-format}.}.  A discussion of the mathematics behind
unnormalized arithmetic is in appendix~\ref{unnorm-math}.

The optimizations that the compiler can perform fall into several categories:

\subsection{Post Multiply Normalization}

When more than two multiplications are performed in a row, the intermediate
normalization of the results between multiplications can be eliminated.
This is because with each multiplication, the mantissa can become
denormalized by at most one bit.  If there are guard bits on the mantissas
to prevent bits from ``falling off'' the end during multiplications, the
normalization can be postponed until after a sequence of several
multiplies\footnote{Using unnormalized numbers for math is not a new idea; a
good example of it is the Control Data CDC 6600, designed by Seymour Cray.
\cite{thornton:cdc6600} The CDC 6600 had all of its instructions performing
unnormalized arithmetic, with a separate {\tt NORMALIZE} instruction.}.

% This is an example of how you would use tgrind to include an example
% of source code; it is commented out in this template since the code
% example file does not exist.  To use it, you need to remove the '%' on the
% beginning of the line, and insert your own information in the call.
%
%\tagrind[htbp]{code/pmn.s.tex}{Post Multiply Normalization}{opt:pmn}

As you can see, the intermediate results can be multiplied together, with no
need for intermediate normalizations due to the guard bit.  It is only at
the end of the operation that the normalization must be performed, in order
to get it into a format suitable for storing in memory\footnote{Note that
for purposed of clarity, the pipeline delays were considered to be 0, and
the branches were not delayed.}.

\subsection{Block Exponent}

In a unoptimized sequence of additions, the sequence of operations is as
follows for each pair of numbers ($m_1$,$e_1$) and ($m_2$,$e_2$).
\begin{enumerate}
  \item Compare $e_1$ and $e_2$.
  \item Shift the mantissa associated with the smaller exponent $|e_1-e_2|$
        places to the right.
  \item Add $m_1$ and $m_2$.
  \item Find the first one in the resulting mantissa.
  \item Shift the resulting mantissa so that normalized
  \item Adjust the exponent accordingly.
\end{enumerate}

Out of 6 steps, only one is the actual addition, and the rest are involved
in aligning the mantissas prior to the add, and then normalizing the result
afterward.  In the block exponent optimization, the largest mantissa is
found to start with, and all the mantissa's shifted before any additions
take place.  Once the mantissas have been shifted, the additions can take
place one after another\footnote{This requires that for n consecutive
additions, there are $\log_{2}n$ high guard bits to prevent overflow.  In
the $\mu$FPU, there are 3 guard bits, making up to 8 consecutive additions
possible.}.  An example of the Block Exponent optimization on the expression
X = A + B + C is given in figure~\ref{opt:be}.

% This is an example of how you would use tgrind to include an example
% of source code; it is commented out in this template since the code
% example file does not exist.  To use it, you need to remove the '%' on the
% beginning of the line, and insert your own information in the call.
%
%\tgrind[htbp]{code/be.s.tex}{Block Exponent}{opt:be}

\section{Integer optimizations}

As well as the floating point optimizations described above, there are
also integer optimizations that can be used in the $\mu$FPU.  In concert
with the floating point optimizations, these can provide a significant
speedup.

\subsection{Conversion to fixed point}

Integer operations are much faster than floating point operations; if it is
possible to replace floating point operations with fixed point operations,
this would provide a significant increase in speed.

This conversion can either take place automatically or or based on a
specific request from the programmer.  To do this automatically, the
compiler must either be very smart, or play fast and loose with the accuracy
and precision of the programmer's variables.  To be ``smart'', the computer
must track the ranges of all the floating point variables through the
program, and then see if there are any potential candidates for conversion
to floating point.  This technique is discussed further in
section~\ref{range-tracking}, where it was implemented.

The other way to do this is to rely on specific hints from the programmer
that a certain value will only assume a specific range, and that only a
specific precision is desired.  This is somewhat more taxing on the
programmer, in that he has to know the ranges that his values will take at
declaration time (something normally abstracted away), but it does provide
the opportunity for fine-tuning already working code.

Potential applications of this would be simulation programs, where the
variable represents some physical quantity; the constraints of the physical
system may provide bounds on the range the variable can take.
\subsection{Small Constant Multiplications}

One other class of optimizations that can be done is to replace
multiplications by small integer constants into some combination of
additions and shifts.  Addition and shifting can be significantly faster
than multiplication.  This is done by using some combination of
\begin{eqnarray*}
a_i & = & a_j + a_k \\
a_i & = & 2a_j + a_k \\
a_i & = & 4a_j + a_k \\
a_i & = & 8a_j + a_k \\
a_i & = & a_j - a_k \\
a_i & = & a_j \ll m \mbox{shift}
\end{eqnarray*}
instead of the multiplication.  For example, to multiply $s$ by 10 and store
the result in $r$, you could use:
\begin{eqnarray*}
r & = & 4s + s\\
r & = & r + r
\end{eqnarray*}
Or by 59:
\begin{eqnarray*}
t & = & 2s + s \\
r & = & 2t + s \\
r & = & 8r + t
\end{eqnarray*}
Similar combinations can be found for almost all of the smaller
integers\footnote{This optimization is only an ``optimization'', of course,
when the amount of time spent on the shifts and adds is less than the time
that would be spent doing the multiplication.  Since the time costs of these
operations are known to the compiler in order for it to do scheduling, it is
easy for the compiler to determine when this optimization is worth using.}.
\cite{magenheimer:precision}

\section{Other optimizations}

\subsection{Low-level parallelism}

The current trend is towards duplicating hardware at the lowest level to
provide parallelism\footnote{This can been seen in the i860; floating point
additions and multiplications can proceed at the same time, and the RISC
core be moving data in and out of the floating point registers and providing
flow control at the same time the floating point units are active. \cite{byte:i860}}

Conceptually, it is easy to take advantage to low-level parallelism in the
instruction stream by simply adding more functional units to the $\mu$FPU,
widening the instruction word to control them, and then scheduling as many
operations to take place at one time as possible.

However, simply adding more functional units can only be done so many times;
there is only a limited amount of parallelism directly available in the
instruction stream, and without it, much of the extra resources will go to
waste.  One process used to make more instructions potentially schedulable
at any given time is ``trace scheduling''.  This technique originated in the
Bulldog compiler for the original VLIW machine, the ELI-512.
\cite{ellis:bulldog,colwell:vliw}  In trace scheduling, code can be
scheduled through many basic blocks at one time, following a single
potential ``trace'' of program execution.  In this way, instructions that
{\em might\/} be executed depending on a conditional branch further down in
the instruction stream are scheduled, allowing an increase in the potential
parallelism.  To account for the cases where the expected branch wasn't
taken, correction code is inserted after the branches to undo the effects of
any prematurely executed instructions.

\subsection{Pipeline optimizations}

In addition to having operations going on in parallel across functional
units, it is also typical to have several operations in various stages of
completion in each unit.  This pipelining allows the throughput of the
functional units to be increased, with no increase in latency.

There are several ways pipelined operations can be optimized.  On the
hardware side, support can be added to allow data to be recirculated back
into the beginning of the pipeline from the end, saving a trip through the
registers.  On the software side, the compiler can utilize several tricks to
try to fill up as many of the pipeline delay slots as possible, as
seendescribed by Gibbons. \cite{gib86}

\chapter{Fisierele clasa}

Fisierele de clasa Java sunt formate din 10
sectiuni\cite{classfile_sections}:

\begin{enumerate}
\item
  Constanta magica.
\item
  Versiunea fisierului.
\item
  Constantele clasei.
\item
  Permisiunile de acces.
\item
  Numele clasei din fisier.
\item
  Numele superclasei.
\item
  Interfetele pe care clasa le implementeaza.
\item
  Campurile clasei.
\item
  Metodele clasei.
\item
  Atribute ale clasei.
\end{enumerate}

In continuare voi da o scurta descriere a formatului sectiunilor.

\section{Sectiunile fiserelor clasa}

\subsection{Magic}

Toate fiserele clasa trebuiesc sa inceapa cu un numar denumit constanta
magica. Acesta este folosit pentru a identifica in mod unic ca acestea
sunt intra-devar fisiere clasa. Numarul magic are o valoare memorabila:
reprezentarea hexadecimala este \texttt{0xCAFEBABE},

\subsection{Versiunea}

Versiunea unui fisier clasa este data de doua valori, versiunea majora
\texttt{M} si versiunea minora \texttt{m}. Versiunea clasei este atunci
reprezentata ca \texttt{M.m}. (e.g., \texttt{45.1}). Aceasta este
folosita pentru a mentine compatibilitatea in cazul modificarilor
masinii virtuale care interpreteaza clasa sau ale compilatorului care o
genereaza.

\subsection{Constantele clasei}

Tabela de constante este locul unde sunt stocate valorile literale
constante ale clasei:

\begin{itemize}
    \item Numere intregi.
    \item Numere cu virgula mobula.
    \item Siruri de caractere, care pot reprezenta la randul lor:
        \begin{itemize}
            \item Nume de clase.
            \item Nume de metode.
            \item Tipuri ale metodelor.
        \end{itemize}
    \item Informatii compuse din datele anterioare:
        \begin{itemize]}
            \item Referinta la o metoda a unei clase.
            \item Referinta la o constanta a unei clase.
        \end{itemize]}
\end{itemize}

Toate celelalte tipuri de date compuse, cum ar fi metodele sau
campurile, vor contine indecsi in tabela de constante.

\subsection{Permisiunile de acces}

Aceste permisiuni constau intr-o masca de bitsi, care reprezeinta
operatiile permise pe aceasta clasa:

\begin{itemize}
    \item daca clasa este publica, si poate fi accesta din afara pachetului acesteia.
    \item daca clasa este finala, si daca poate fi extinsa.
    \item daca invocarea metodelor din superclasa sa fie tratata special.
    \item daca este de fapt o interfata, si nu o clasa.
    \item daca este o clasa abstracta si nu poate fi instatiata.
\end{itemize}

\subsection{Clasa curenta}

Reprezinta un indice in tabela de constante, unde sunt stocate
informatii despre clasa curenta.

\subsection{Clasa super}

Reprezinta un indice in tabela de consatante, cu informatii despre clasa
din care a mostenit clasa curenta. Daca este 0, inseamna ca clasa
curenta nu mosteneste nimic: singura clasa fara o superclasa este clasa
Object.

E.g. pentru

\begin{lstlisting}[language=Java]
public class MyClass extends S implements I
\end{lstlisting}

Indicele corespunde lui \texttt{S}.

\subsection{Interfetele}

Reprezinta o colectie de indici in tabela de constante. Fiecare valoare
de la acei indici reprezinta o interfata implementata in mod direct de
catre clasa curenta. Interfetele apar in ordinea declarata in fisierele
java.

E.g. pentru

\begin{lstlisting}[language=Java]
class MyClass extends S implements I1, I2
\end{lstlisting}

Primul indice ar corespunde lui \texttt{I1}, iar al doilea lui
\texttt{I2}.

\subsection{Campurile}\label{campurile}

Reprezinta informatii despre campurile (eng. fields) clasei:
\begin{itemize}
    \item Permisiunile de acces: daca este public sau privat, etc.
    \item Numele campului.
    \item Tipul campului.
    \item Alte atribute: daca este deprecat, daca are o valoare constanta, etc.
\end{itemize}

\subsection{Metodele}\label{metodele}

Reprezinta informtii despre toate metodele clasei, si include si
constructorii:

\begin{itemize}
    \item Permisiuni de acces: daca este public sau privat, daca este finala, daca este abstracta.
    \item Numele metodei.
    \item Tipul metodei.
    \item In caz ca nu este abstracta, byte codul metodei.
    \item Alte atribute:
        \begin{itemize}
            \item Ce exceptii poate arunca.
            \item Daca este deprecata.
        \end{itemize}
\end{itemize}

Codul metodei este partea cea mai importanta, iar formatul acestuia
urmeaza sa fie detaliat ulterior.

\subsection{Atributele}

Reprezinta alte informatii despre clasa, cum ar fi:
\begin{itemize}
    \item Clasele definite in interiorul acesteia.
    \item In caz ca este o clasa anonima sau definita local, metoda in care este definita.
    \item Numele fisierul sursa din care a fost compilata clasa.
\end{itemize}

In continuare, voi descrie din punct de vedere tehnic tipurile de date
intalnite in fisierele de clasa:

\subsection{Tipurile de baza}

In formatul fisierelor clasa exista trei tipuri de baza, toate bazate pe
intregi. In caz ca un intreg are mai multi octeti, acestia au ordinea de
\texttt{big-endian}: cel mai semnificativ octet va fi mereu primul in
memorie.

\begin{longtable}[]{@{}ccc@{}}
\toprule
Nume & Semantica & Echivalentul in C\tabularnewline
\midrule
\endhead
\texttt{u1} & intreg pe un octet, fara semn & \texttt{unsigned\ char}
sau \texttt{uint8\_t}\tabularnewline
\texttt{u2} & intreg pe doi octeti, fara semn & \texttt{unsigned\ short}
sau \texttt{uint16\_t}\tabularnewline
\texttt{u4} & intreg pe un octet, fara semn & \texttt{unsigned\ int} sau
\texttt{uint32\_t}\tabularnewline
\bottomrule
\end{longtable}

In codul sursa al proiectului, acestea sunt tratate astfel:

\begin{lstlisting}[language=C++]
using u1 = uint8_t;
using u2 = uint16_t;
using u4 = uint32_t;
\end{lstlisting}

\subsection{Tipuri de date compuse}

\subsubsection{cp\_info}

Fiecare constanta din tabela de constante incepe cu o eticheta de 1
octet, care reprezinta datele si tipul structurii. Continutul acesteia
variaza in functie de eticheta, insa indiferent de eticheta, continutul
trebuie sa aiba cel putin 2 octeti.

Aproape toate tipurile de constante ocupa un singur slot in tabela.
Din motive istorice, unele constante ocupa doua sloturi.

Totodata, tot din motive istorice, tabela este indexata de la 1, si nu
de la 0, cum sunt celelalte.

\subparagraph{Tipurile de constante}\label{tipurile-de-constante}

\texttt{CONSTANT\_Class}

Corespunde valorii etichetei de 7 si contine un indice spre un alt camp
in tabela de constante, de tipul \texttt{CONSTANT\_Utf8} - un sir de
caractere. Acel sir de caractere va contine numele clasei.

\texttt{CONSTANT\_Fieldref}

Corespunde valorii etichetei de 9 si contine o referinta spre campul
unei clase. Referinta conta in doi indici, amandoi care arata spre
tabela de contante. Primul indice arata spre o constanta
\texttt{CONSTANT\_Class}, care reprezinta clasa sau interfata careia
apartine metoda. Al doilea indice arata spre o constanta
\texttt{CONSTANT\_NameAndType}, care contine informatii despre numele si
tipul campului.

\texttt{CONSTANT\_Methodref}

Corespunde valorii etichetei de 10 si contine o referinta spre metoda
unei clase. Are o structura identica cu \texttt{CONSTANT\_Fieldref},
doar ca primul indice arata neaparat spre o clasa, in timp ce al doilea
indice arata spre numele si tipul metodei.

\texttt{CONSTANT\_InterfaceMethodref}

Corespunde valorii etichetei de 11 si contine o refereinta spre metoda
unei interfete. Are o structura identica cu
\texttt{CONSTANT\_Methodref}, doar ca primul indice arata spre o
interfata.

\texttt{CONSTANT\_String}

Corespunde valorii etichetei de 8 si reprezinta un sir de caractere.
Contine un indice, catre o structura de tipul \texttt{CONSTANT\_Utf8}.

\texttt{CONSTANT\_Integer}

Corespunde valorii etichetei de 3 si contine un intreg pe 4 octeti.

\texttt{CONSTANT\_Float}

Corespunde valorii etichetei de 4 si contine un numar cu virgula mobila
pe 4 octeti.

\texttt{CONSTANT\_Long}

Corespunde valorii etichetei de 5 si contine un intreg pe 8 octeti. Din
motive istorice, ocupa 2 spatii in tabela de constante.

\texttt{CONSTANT\_Double}

Corespunde valorii etichetei de 6 si contine un numar cu virgula mobila
pe 8 octeti. Din motive istorice, ocupa 2 spatii in tabela de constante.

\texttt{CONSTANT\_NameAndType}

Corespunde valorii etichetei de 12. Descrie numele si tipul unui camp
sau al unei metode, fara informatii despre clasa. Contine doi indici,
amandoi catre structuri de tipul \texttt{CONSTANT\_Utf8}. Primul
reprezinta numele, iar al doilea tipul.

\texttt{CONSTANT\_Utf8}

Corespunde valorii etichetei de 1. Reprezinta un sir de caractere
encodat in formatul UTF-8. Contine un intreg \texttt{length}, de tipul
\texttt{u2}, si apoi \texttt{length} octeti care descriu sirul in sine.
Din cauza ca este encodat ca UTF-8, un singur caracter poate fi format
din mai multi octeti.

\texttt{CONSTANT\_MethodHandle}

Corespunde valorii etichetei de 15 si contine o referinte catre un camp,
o metoda de clasa, sau o metoda de interfata.

\texttt{CONSTANT\_MethodType}

Corespunde valorii etichetei de 16 si contine un indice catre o
constanta \texttt{CONSTANT\_UTf8}, ce reprezinta tipul unei metode.

\texttt{CONSTANT\_InvokeDynamic}

Corespunde valorii etichetei de 18 si este folosit de catre \texttt{JVM}
pentru a invoka o metoda polimorfica.

In cod \texttt{C++}, am reprezentat \texttt{cp\_info} astfel:

\begin{lstlisting}[language=C++]
struct cp_info {
    enum class Tag : u1 {
        CONSTANT_Class = 7,
        CONSTANT_Fieldref = 9,
        CONSTANT_Methodref = 10,
        CONSTANT_InterfaceMethodref = 11,
        CONSTANT_String = 8,
        CONSTANT_Integer = 3,
        CONSTANT_Float = 4,
        CONSTANT_Long = 5,
        CONSTANT_Double = 6,
        CONSTANT_NameAndType = 12,
        CONSTANT_Utf8_info = 1,
        CONSTANT_MethodHandle = 15,
        CONSTANT_MethodType = 16,
        CONSTANT_InvokeDynamic = 18,
    };

    Tag tag;
    std::vector<u1> data;
};
\end{lstlisting}

Iar structurile folosite pentru obiectivul propus au fost reprezentate
astfel:

\begin{lstlisting}[language=C++]
struct CONSTANT_Methodref_info {
    cp_info::Tag tag;
    u2 class_index;
    u2 name_and_type_index;
};
struct CONSTANT_Class_info {
    cp_info::Tag tag;
    u2 name_index;
};
struct CONSTANT_NameAndType_info {
    cp_info::Tag tag;
    u2 name_index;
    u2 descriptor_index;
};
\end{lstlisting}

\paragraph{\texorpdfstring{\texttt{field\_info}}{field\_info}}\label{field_info}

Fiecare camp din cadrul unei clase este reprezentat printr-o structura
de tipul \texttt{field\_info}.

In cod \texttt{C++}, aceasta structura a fost reprezentata astfel:

\begin{lstlisting}[language=C++]
struct field_info {
    u2 access_flags;
    u2 name_index;
    u2 descriptor_index;
    u2 attributes_count;
    std::vector<attribute_info> attributes;
};
\end{lstlisting}

Unde:
\begin{itemize}
    \item \texttt{name\_index} este o intrare in tabela de constante unde se afla o constanta de tipul \texttt{CONSTANT\_Utf8}.
    \item \texttt{descriptor\_index} arata spre o constanta de tipul \texttt{CONSTANT\_Utf8} si reprezinta tipul campului.
\end{itemize}

\paragraph{\texorpdfstring{\texttt{method\_info}}{method\_info}}\label{method_info}

Fiecare metoda a unei clase/interfete este descrisa prin aceasta
structura.

In cod \texttt{C++}, am implementat-o asa:

\begin{lstlisting}[language=C++]
struct method_info {
    u2 access_flags;
    u2 name_index;
    u2 descriptor_index;
    u2 attributes_count;
    std::vector<attribute_info> attributes;
};
\end{lstlisting}

Unde \texttt{name\_index} si \texttt{descriptor\_index} au aceeasi
interpretare ca si la \texttt{field\_info}.

Daca metoda nu este abstracta, atunci in vectorul \texttt{attributes} se
va gasi un attribut de tipul \texttt{Code}, care contine bytecode-ul
corespunzator acestei metode.

\paragraph{\texorpdfstring{\texttt{attribute\_info}}{attribute\_info}}\label{attribute_info}

In \texttt{C++}, a fost implementata astfel:

\begin{lstlisting}[language=C++]
struct attribute_info {
    u2 attribute_name_index;
    u4 attribute_length;
    std::vector<u1> info;
};
\end{lstlisting}

Numele atributului determina modul in care octetii din vectorul
\texttt{info} sunt interpretati. Pentru intentiile noastre, atributul de
interes este cel de cod:

\subparagraph{\texorpdfstring{\texttt{Code\_attribute}}{Code\_attribute}}\label{code_attribute}

\begin{lstlisting}[language=C++]
struct Code_attribute {
    u2 attribute_name_index;
    u4 attribute_length;

    u2 max_stack;
    u2 max_locals;

    u4 code_length;
    std::vector<u1> code;

    u2 exception_table_length;
    struct exception {
        u2 start_pc;
        u2 end_pc;
        u2 handler_pc;
        u2 catch_type;
    };
    std::vector<exception> exception_table; // of length exception_table_length.
    u2 attributes_count;
    std::vector<attribute_info> attributes; // of length attributes_count.
};
\end{lstlisting}

Aceasta structura este piesa centrala a lucrarii. In continuare, o voi
descrie detaliat:

\begin{itemize}
\tightlist
\item
  \texttt{max\_stack}: Reprezinta adancimea maxima a stivei masinii
  virtuale cand aceasta bucata de cod este interpretata.
\item
  \texttt{max\_locals}: Reprezinta numarul maxim de variabile locale
  alocate in acelasi timp cand aceasta bucata de cod este interpretata.
\item
  \texttt{code}: Codul metodei.
\item
  \texttt{exception\_table}: Exceptiile pe care le poate arunca metoda.
\end{itemize}

\texttt{Code}

Vectorul \texttt{code} din cadrul atributului \texttt{Code} reprezinta
bytecode-ul propriu-zis al metodei.

Acest vector contine instructiunile care sunt executate de catre masina
virtuala.

JVM-ul ruleaza ca o masina cu stiva, iar toate instructiunile opereaza
pe aceasta stiva. Reultatul rularii unei instructiuni este modificarea
stivei: scoaterea si adaugarea de elemente in varful acesteia.

Instructiunile au in general formatul \cite{instruction_format}:

\begin{verbatim}
nume_instr
operand1
operand2
...
\end{verbatim}

cu un numar variabil de operanzi, prezenti in mod explicit in vectorul
de \texttt{cod}.

Fiecarui instructiuni ii corespunde un octet, denumit opcode. Fiecare
operand este fie cunoscut la compilare, fie calculat in mod dinamic la
rulare.

Cele mai multe operatii nu au niciun operand dat in mod explicit la
nivelul instructiunii: ele lucreaza doar cu valorile din varful stivei
la momentul executarii codului.

De exemplu:

Instructiunea \texttt{imul} are octetul \texttt{104} sau \texttt{0x68}.
Acestea da pop la doua valori din varful stivei: \texttt{value1} si
\texttt{value2}. Amandoua valorile trebuie sa fie de tipul \texttt{int}.
Rezultatul este inmultirea celor doua valori:
\texttt{result\ =\ value1\ *\ value2}, si este pus in varful stivei.

Dintre cele peste o suta de instructiuni, noi suntem preocupati doar de
5 dintre acestea: cele care au de a face cu invocarea unei metode.

invokedynamic

Format:

\begin{verbatim}
invokedynamic
index1
index2
0
0
\end{verbatim}

Opcode-ul corespunzator acestei instructiuni este \texttt{186} sau
\texttt{0xba}.

Index1 si index2 sunt doi octeti sunt compusi in

\begin{Shaded}
\begin{Highlighting}[]
\NormalTok{index = (index1 << }\DecValTok{8}\NormalTok{) | index2}
\end{Highlighting}
\end{Shaded}

Indicele compus reprezinta o intrare in tabela de constante. La locatia
respectiva trebuie sa se afle o structura de tipul
\texttt{CONSTANT\_MethodHandle}

invokeinterface

Format:

\begin{verbatim}
invokeinterface
index1
index2
count
0
\end{verbatim}

Opcode-ul corespunzator este \texttt{185} sau \texttt{0xb9}.
\texttt{index1} si \texttt{index2} sunt folositi, in mod similar ca la
\texttt{invokedynamic}, pentru a construi un \texttt{indice} in tabela
de constante.

La pozitia respectiva in tabela, trebuie sa se regaseasca o strutura de
tipul \texttt{CONSTANT\_Methodref}.

\texttt{count} trebuie sa fie un octet fara semn diferit de 0. Acest
operand descrie numarul argumentelor metodei, si este necesar din motive
istorice: aceasta informatie poate fi dedusa din tipul metodei.

TODO(ericpts): add resolution order.

invokespecial

Format:

\begin{verbatim}
invokespecial
index1
index2
\end{verbatim}

Opcode-u corespunzator este \texttt{183} sau \texttt{0xb7}. La fel ca la
\texttt{invokeinterface}, este format un indice in tabela de constante,
catre o structura \texttt{CONSTANT\_Methodref}.

Aceasta instructiune este folosita pentru a invoca constructorii
claselor.

invokestatic

Format:

\begin{verbatim}
invokestatic
index1
index1
\end{verbatim}

Opcode-ul corespunzator este \texttt{184} sau \texttt{0xb8}.
Instructiunea este invocata pentru a invoke o metoda statica a unei
clase.

La fel ca la \texttt{invokeinterface}, este construit un indice compus,
si folosit pentru a indexa tabela de constante.

invokevirtual

Format:

\begin{verbatim}
invokevirtual
index1
index1
\end{verbatim}

Opcode-ul corespunzator este \texttt{182} sau \texttt{0xb6}, iar
interpretarea este la fel ca la \texttt{invokeinterface}.

Aceasta este cea mai comuna instructiune de invocare de functii.

Dupa ce numele si tipul metodei, cat si clasa \texttt{C} de care
apartine aceasta sunt rezolvate, masina virtuala cauta metoda respectiva
in clasa referentiata. In caz ca o gaseste, cautarea se termina. In caz
negativ, JVM va continua cautarea recursiv din superclasa lui
\texttt{C}.

In \texttt{C++}, am reprezentat aceste instructiuni de interes astfel:

\begin{lstlisting}[language=C++]
enum class Instr {
    invokedynamic = 0xba,
    invokeinterface = 0xb9,
    invokespecial = 0xb7,
    invokestatic = 0xb8,
    invokevirtual = 0xb6,
};
\end{lstlisting}

\subsection{ClassFile}\label{classfile}

Folosind definitiile anterioare, putem descrie un fisier de clasa binar
in C++:


\begin{lstlisting}[language=C++]
struct ClassFile {
    u4 magic; // Should be 0xCAFEBABE.

    u2 minor_version;
    u2 major_version;

    u2 constant_pool_count;
    std::vector<cp_info> constant_pool;

    u2 access_flags;

    u2 this_class;
    u2 super_class;

    u2 interface_count;
    std::vector<interface_info> interfaces;

    u2 field_count;
    std::vector<field_info> fields;

    u2 method_count;
    std::vector<method_info> methods;

    u2 attribute_count;
    std::vector<attribute_info> attributes;
};
\end{lstlisting}

\section{Studiu de caz}

Pentru a vedea un fisier clasa analizat in detaliu, uitati-va la appendix-ul studiu de caz.


\appendix

\section{Appendix - Studiu de caz}

In continuare, voi exemplica structura unui fisier clasa cu un exemplu.

Codul Java este urmatorul:

\begin{lstlisting}[language=Java]
public class Main {
    public static void main(String[] args) {
        System.out.println("project1 - hello world");
        foo();
    }

    public static void foo() {
        System.out.println("project1 - foo()");
    }
}
\end{lstlisting}

Compilatorul folosit este \texttt{openjdk-11}. Clasa a fost utilizata
folosind utilitarul \texttt{javap} \cite{javap}, care este de asemenea inclus
in pachetul \texttt{openjdk-11}.

In primul rand, tabela de constante:

\begin{lstlisting}
Constant pool:
   #1 = Methodref          #8.#18         // java/lang/Object."<init>":()V
   #2 = Fieldref           #19.#20        // java/lang/System.out:Ljava/io/PrintStream;
   #3 = String             #21            // project1 - hello world
   #4 = Methodref          #22.#23        // java/io/PrintStream.println:(Ljava/lang/String;)V
   #5 = Methodref          #7.#24         // Main.foo:()V
   #6 = String             #25            // project1 - foo()
   #7 = Class              #26            // Main
   #8 = Class              #27            // java/lang/Object
   #9 = Utf8               <init>
  #10 = Utf8               ()V
  #11 = Utf8               Code
  #12 = Utf8               LineNumberTable
  #13 = Utf8               main
  #14 = Utf8               ([Ljava/lang/String;)V
  #15 = Utf8               foo
  #16 = Utf8               SourceFile
  #17 = Utf8               Main.java
  #18 = NameAndType        #9:#10         // "<init>":()V
  #19 = Class              #28            // java/lang/System
  #20 = NameAndType        #29:#30        // out:Ljava/io/PrintStream;
  #21 = Utf8               project1 - hello world
  #22 = Class              #31            // java/io/PrintStream
  #23 = NameAndType        #32:#33        // println:(Ljava/lang/String;)V
  #24 = NameAndType        #15:#10        // foo:()V
  #25 = Utf8               project1 - foo()
  #26 = Utf8               Main
  #27 = Utf8               java/lang/Object
  #28 = Utf8               java/lang/System
  #29 = Utf8               out
  #30 = Utf8               Ljava/io/PrintStream;
  #31 = Utf8               java/io/PrintStream
  #32 = Utf8               println
  #33 = Utf8               (Ljava/lang/String;)V
\end{lstlisting}

In acest format, namespace-urile imbricate sunt reprezentate prin
\texttt{/}.

Informatii despre clasa:

\begin{lstlisting}
Classfile Main.class
  Last modified May 28, 2018; size 520 bytes
  MD5 checksum 248b729dfe4b4bc8da895944d30fdc28
  Compiled from "Main.java"
public class Main
  minor version: 0
  major version: 55
  flags: (0x0021) ACC_PUBLIC, ACC_SUPER
  this_class: #7                          // Main
  super_class: #8                         // java/lang/Object
  interfaces: 0, fields: 0, methods: 3, attributes: 1
\end{lstlisting}

Constructorul clasei:

\begin{lstlisting}
{
  public Main();
    descriptor: ()V
    flags: (0x0001) ACC_PUBLIC
    Code:
      stack=1, locals=1, args_size=1
         0: aload_0
         1: invokespecial #1                  // Method java/lang/Object."<init>":()V
         4: return
      LineNumberTable:
        line 1: 0
\end{lstlisting}

Metoda \texttt{main(String{[}{]}\ args)}:

\begin{lstlisting}
public static void main(java.lang.String[]);
descriptor: ([Ljava/lang/String;)V
flags: (0x0009) ACC_PUBLIC, ACC_STATIC
Code:
  stack=2, locals=1, args_size=1
     0: getstatic     #2                  // Field java/lang/System.out:Ljava/io/PrintStream;
     3: ldc           #3                  // String project1 - hello world
     5: invokevirtual #4                  // Method java/io/PrintStream.println:(Ljava/lang/String;)V
     8: invokestatic  #5                  // Method foo:()V
    11: return
  LineNumberTable:
    line 3: 0
    line 4: 8
    line 5: 11
\end{lstlisting}

Metoda \texttt{foo()}:

\begin{lstlisting}
public static void foo();
descriptor: ()V
flags: (0x0009) ACC_PUBLIC, ACC_STATIC
Code:
  stack=2, locals=0, args_size=0
     0: getstatic     #2                  // Field java/lang/System.out:Ljava/io/PrintStream;
     3: ldc           #6                  // String project1 - foo()
     5: invokevirtual #4                  // Method java/io/PrintStream.println:(Ljava/lang/String;)V
     8: return
  LineNumberTable:
    line 8: 0
    line 9: 8
\end{lstlisting}

%% \chapter{Dummy Appendix}

You can defer lengthy calculations that would otherwise only interrupt
the flow of your thesis to an appendix.


\backmatter

\bibliographystyle{plain}
\bibliography{refs}

\end{document}
